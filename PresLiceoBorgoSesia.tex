%~~~~~~~~~~~~~~~~~~~~~~~~~~~~~~~~~~~~~~~~~~~~~~~~~~~~~~~~~~~~~~~~~~~~~~~~~~~~~~~
%Presentazione: Ambiente e Salute
%Luogo: Borgo Sesia Liceo Scientifico
%Autore: Pasquale Scordino
%Data: 20 Luglio 2018
%~~~~~~~~~~~~~~~~~~~~~~~~~~~~~~~~~~~~~~~~~~~~~~~~~~~~~~~~~~~~~~~~~~~~~~~~~~~~~~~
\documentclass{beamer}
\usetheme{CambridgeUS}
\usecolortheme{beaver}
\usefonttheme[onlymath]{serif}

%Settaggio colori e forma degli elenchi puntati
\setbeamercolor{itemize item}{fg=darkred!80!black}
\setbeamercolor{itemize subitem}{fg=orange}
\setbeamercolor{itemize subsubitem}{fg=cyan}

\setbeamertemplate{itemize item}[triangle]
\setbeamertemplate{itemize subitem}[circle]
\setbeamertemplate{itemize subsubitem}[triangle]

%Settaggio colore testo e sfondo del titolo del blocco
\setbeamercolor{block title}{fg=white,bg=darkred!80!black}
%~~~~~~~~~~~~~~~~~~~~~~~~~~~~~~~~~~~~~~~~~~~~~~~~~~~~~~~~~~~~~~~~~~~~~~~~~~~~~~~~
%Pacchetti usati
\usepackage[italian]{babel}
\usepackage[latin1]{inputenc}
\usepackage{times}
\usepackage[T1]{fontenc}
\usepackage{graphics}
\usepackage{subfigure}
\usepackage{multimedia}
\usepackage{pifont,xcolor} % http://ctan.org/pkg/{pifont,xcolor}
\usepackage{fancybox}
\usepackage{textpos}
\usepackage{amsmath}
\usepackage{xcolor}
%~~~~~~~~~~~~~~~~~~~~~~~~~~~~~~~~~~~~~~~~~~~~~~~~~~~~~~~~~~~~~~~~~~~~~~~~~~~~~~~~~
%Prima slide intestazione
\title[Ambiente e Salute]{\textbf{Introduzione al concetto di Inquinamento}}

\author[P. Scordino\hspace{0.45cm}{p.scordino@arpa.piemonte.it}]{P. Scordino}
\institute{ARPA Piemonte}

\titlegraphic{

  \includegraphics[width=2cm,height=2cm,keepaspectratio]{Images/Arpacolori.jpg}%
  \hspace*{8.20cm}
  \includegraphics[width=2cm,height=2cm,keepaspectratio]{Images/logo_SNPA_COL.jpg}%

}
%~~~~~~~~~~~~~~~~~~~~~~~~~~~~~~~~~~~~~~~~~~~~~~~~~~~~~~~~~~~~~~~~~~~~~~~~~~~~~~~~~
%Inizio 
\begin{document}

\section*{Inizio}
\begin{frame}
\titlepage
\end{frame}

\section*{Sommario}
\begin{frame}
	\frametitle{Sommario}
	\begin{center}
		\begin{block}{Argomenti trattati}
		\begin{itemize}
			\item \textbf{L'ambiente}
			\item \textbf{Il rapporto Uomo-Ambiente}
			\item \textbf{Il concetto di inquinamento}
			\item \textbf{L'inquinamento atmosferico}
			\item \textbf{L'inquinamento da pesticidi/erbicidi}
		\end{itemize}
		\end{block}
	\end{center}
\end{frame}

%%%%%%%%%%%%%%%%%%%%%%%%%%%%%%%%%%%%%%%%%%%%%%%%%%%%%%%%%%%%%%%%%%%%%%%%%%%%%%%%%%%%%%%%%%%%%%%%%%%%%%%%%%%%%%%%%%%%%%%%
%
%  Slides - Biosfera
%
%%%%%%%%%%%%%%%%%%%%%%%%%%%%%%%%%%%%%%%%%%%%%%%%%%%%%%%%%%%%%%%%%%%%%%%%%%%%%%%%%%%%%%%%%%%%%%%%%%%%%%%%%%%%%%%%%%%%%%%%

\section*{Biosfera}
\begin{frame}
	\frametitle{L'ambiente}
	La \textbf{biosfera} conosciuta anche come ecosfera � costituita da tutti gli \textbf{ecosistemi} presenti sulla terra. Puo essere considerata come la zona vitale, � un \textbf{sistema chiuso} e largamente \textbf{autoregolato}. 
	\begin{center}
		\begin{block}{BIOSFERA}
			\begin{figure}
			\href{Images/Video/biosfera.mp4}{\includegraphics[scale=0.19]{Images/Immagini/BiosphereLg.png}}
			\end{figure}
		\end{block}
	\end{center}
\end{frame}

\section*{L'ecosistema}
\begin{frame}
	\frametitle{L'ecosistema}
	L'ecosistema � l'unit� ecologica fondamentale, formata da una comunit� di organismi viventi in una determinata area \textbf{(biocenosi)} e dallo specifico ambiente fisico \textbf{(biotopo)}, con il quale gli organismi sono legati da complesse interazioni e scambi di energia e di materia.
	
	Un ecosistema comprende diversi \textbf{habitat} e differenti \textbf{nicchie ecologiche}.
	
	L'habitat � il luogo fisico dove un animale o una pianta vivono normalmente, in genere caratterizzato da una forma vegetale o da un aspetto fisico dominante.
	
	La nicchia ecologica � il ruolo ecologico, o "funzione", che ogni specie occupa all'interno di un habitat, cio� � uno spazio che include tutti gli aspetti dell'esistenza di quella specie. Per esempio, una nicchia ecologica � definita dalle esigenze alimentari, dalle abitudini di vita e dalle interazioni della specie considerata con altre specie, oltre che dalle condizioni climatiche e chimico-fisiche.
\end{frame}

\begin{frame}
\frametitle{Schema semplificato di ecosistema}
	\begin{center}
		\begin{block}{Schema}
	 		\begin{figure}
			\includegraphics[scale=1.1]{Images/Immagini/ecosistema.png}		
			\end{figure}
		\end{block}
	\end{center}
\end{frame}

\begin{frame}
\frametitle{L'uomo e la biosfera}
La \textbf{comunit� umana} vive nella biosfera e della biosfera, fa parte integrante.
Gli uomini primitivi facevano una vita di continuo spostamento (nomadismo), vivevano di caccia e dei frutti della natura.
La mobilit� era funzione delle esigenze alimentari per la sopravvivenza e per la conservazione e lo sviluppo della specie.
	\begin{center}
	\begin{block}{Rapporto Uomo - Ambiente}
	
		\begin{figure}
	\includegraphics[scale=0.30]{Images/Immagini/biospherecollapse.jpg}	
		\end{figure}
	\end{block}
	\end{center}
\end{frame}

\begin{frame}
\frametitle{L'uomo e la biosfera}
Le prime societ� che hanno sviluppato l'agricoltura e l'allevamento del bestiame, hanno prodotto delle trasformazioni.

La scelta delle specie vegetali da coltivare e delle specie animali da allevare ha comportato aspetti di selezione. 
La simbiosi fra l'uomo e le specie scelte consiste nel fatto che quest'ultime ricevono protezione, cura e nutrimento, a loro volta esse forniscono all'uomo le risorse necessarie per la sua vita e per il suo sviluppo.

Quando l'uomo interviene nei processi di riproduzione e di sviluppo dei vegetali, attraverso la selezione delle sementi, gli incroci, ecc. aumenta la produttivit� dell'agricoltura ma attua un processo di asservimento della Natura.

Lo stesso succede allevando gli animali oppure modificando il territorio costruendo strade, canali, ecc.
\end{frame}

\begin{frame}
\frametitle{Concetto di inquinamento}
\begin{center}
	\small{Si definisce \alert{inquinamento} ogni modificazione della biosfera, dovuta all'introduzione
	nella stessa di una o pi\'u sostanze  in quantit� e con caratteristiche tali da ledere o da 
	costituire un pericolo per la salute umana o per la qualit� dell'ambiente oppure tali da ledere i
	beni materiali o compromettere gli usi legittimi dell'ambiente.}
\end{center}
\begin{center}
	\begin{block}{Schema}	
		\begin{figure}
			\includegraphics[scale=0.15]{Images/Immagini/inquinamento.jpeg}	
		\end{figure}
	\end{block}
\end{center}
\end{frame}

%%%%%%%%%%%%%%%%%%%%%%%%%%%%%%%%%%%%%%%%%%%%%%%%%%%%%%%%%%%%%%%%%%%%%%%%%%%%%%%%%%%%%%%%%%%%%%%%%%%%%%%%%%%%%%%%%%%%%%%%
%
%  Slides - Inquinamento Atmosferico
%
%%%%%%%%%%%%%%%%%%%%%%%%%%%%%%%%%%%%%%%%%%%%%%%%%%%%%%%%%%%%%%%%%%%%%%%%%%%%%%%%%%%%%%%%%%%%%%%%%%%%%%%%%%%%%%%%%%%%%%%%

\section*{Inquinamento atmosferico - Struttura atmosfera}
\begin{frame}
\frametitle{Inquinamento atmosferico}
\begin{block}{Struttura dell'atmosfera}
	\vspace{3mm}
	\begin{columns}[T]
		\begin{column}{.45\textwidth}
			\includegraphics[scale=0.35]{Images/Immagini/atmosfera_2.jpg}
		\end{column}
		\begin{column}{0.45\textwidth}
			\vspace{5mm}
			La troposfera, considerando le masse d'aria in movimento, pu� essere suddivisa in:
			\begin{itemize}
				\item Troposfera libera.
				\item Strato limite planetario $50 - 1000$ $m$ (PBL) 
			\end{itemize}		   
		\end{column}	
	\end{columns}
\end{block}
\end{frame}

\section*{Inquinamento atmosferico - Inversione termica}
\begin{frame}
\frametitle{Inquinamento atmosferico}
\begin{block}{Inversione termica}
Foto dalla alta valle del Po alle ore 17.50 del 5 Agosto 2005
\begin{center}
	\begin{figure}
		\includegraphics[scale=0.09]{Images/Immagini/inversione.jpg}
	\end{figure}
\end{center}	
\end{block}
\end{frame}

\begin{frame}
\frametitle{Inquinamento atmosferico}
\begin{block}{Inversione termica}
	Rappresentazione del fenomeno dell'inversione termica
	\begin{center}
		\begin{figure}
			\includegraphics[scale=0.26]{Images/Immagini/inversionequota.png}
		\end{figure}
	\end{center}	
\end{block}
\end{frame}

\section*{Inquinamento atmosferico - Composizione chimica}
\begin{frame}
\frametitle{Inquinamento atmosferico}
\begin{block}{Composizione chimica dell'atmosfera}
	\begin{center}
		\begin{figure}
			\includegraphics[scale=0.6]{Images/Immagini/compatm.png}
		\end{figure}
	\end{center}	
\end{block}
\end{frame}

\begin{frame}
\frametitle{Inquinamento atmosferico}
\begin{block}{La chimica della troposfera}
	La chimica dell'atmosfera � guidata da una specie chimica chiamata spazzino atmosferico: \textbf{Il radicale ossidrile.}
	
	\vspace{4mm}
	
	In bassa troposfera e in assenza di ossidi di azoto e composti organici, sia di origine naturale che antropica, avvengono le seguenti reazioni chimiche di fotolisi dell'ozono:
	
	\vspace{4mm}
	
	\hspace{4mm} $O_{3} + \alert{h\nu} \longrightarrow O_{2} + O(^{1}D)$ \hspace{16mm} $(\lambda \leq 335$ $nm)$\\
	\hspace{4mm} $O(^{1}D) + M \longrightarrow O(^{3}P) + M$ \hspace{12mm} $(M = N_{2}, O_{2})$\\
	
	\vspace{4mm}
	
	\hspace{4mm} $O(^{3}P) + O_{2} + M \longrightarrow O_{3} + M$ \hspace{9mm} $(M = aria)$\\
	\hspace{4mm} $O(^{1}D) + H_{2}O \longrightarrow 2OH^{.}$

\end{block}
\end{frame}

\begin{frame}
\frametitle{Inquinamento atmosferico}
\begin{block}{La chimica della troposfera}
In atmosfera povera di $NO_x$ ma con la presenza di $VOC_s$ (Composti organici volatili) di origine naturale le
reazioni chimiche che avvengono sono le seguenti:
\vspace{5mm}
\begin{columns}[c]  
	\column{2.2in}
	$HO_{2}^{.} + O_{3} \rightarrow OH^{.} + 2O_{2}$\\
	\vspace{2mm}
	$HO_{2}^{.} + HO_{2}^{.} \rightarrow H_{2}O_{2} + O_{2}$\\
	\vspace{2mm}
	$RO_{2}^{.} + HO_{2}^{.} \rightarrow ROOH + O_{2}$\\
	\vspace{2mm}
	$H_{2}O_{2} + \alert{h\nu} (\lambda < 370 nm) \rightarrow 2OH^{.}$\\
	\column{2in}
	\setbeamercolor{postut}{fg=black,bg=cyan}
	\begin{beamerboxesrounded}[upper=postit ,lower=postut ,shadow=true]{}
		I perossidi organici formatisi si depositano sulle superfici o si disciolgono 
		nell'aereosol acquoso. L'acqua ossigenata, invece, fotolizza dando altri radicali ossidrili.
	\end{beamerboxesrounded}
	\vspace{5mm}
\end{columns}
\end{block}
\end{frame}

\begin{frame}
\frametitle{Inquinamento atmosferico}
\begin{block}{La chimica della troposfera}
In Stratosfera l'ozono si forma attraverso il ciclo di Chapman:
\vspace{8mm}
\begin{columns}[c]  
	\column{2in}
	$O_{2} + \alert{h\nu} (\lambda \leq 290 nm) \rightarrow 2O^{.}$\\
	\vspace{2mm}
	$O^{.} + O_{2} + M \rightarrow O_{3}$\\
	\vspace{2mm}
	$O^{.} + O_{3} \rightarrow 2O_{2}$\\
	\vspace{2mm}
	$O_{3} + \alert{h\nu} \rightarrow O^{.} + O_{2}$\\
	\column{2in}
	\setbeamercolor{postut}{fg=black,bg=cyan}
	\begin{beamerboxesrounded}[upper=postit ,lower=postut ,shadow=true]{}
		Il ciclo di Chapman � stato disturbato da specie chimiche persistenti di origine antropica
		che arrivate in stratosfera sono state fotolizzate ed il loro prodotto ha perturbato il ciclo.
	\end{beamerboxesrounded}
	\vspace{5mm}
\end{columns}
\end{block}
\end{frame}

\begin{frame}
\frametitle{Inquinamento atmosferico}
\begin{block}{Assorbimento dello spettro solare dall'atmosfera}
	\begin{center}
		\begin{figure}
			\includegraphics[scale=0.3]{Images/Immagini/spettro.jpg}
		\end{figure}
	\end{center}	
\end{block}
\end{frame}


\begin{frame}
\frametitle{Inquinamento atmosferico}
L'introduzione di sostanze di origine antropica, ma anche naturale, oltre una certa soglia provoca lo squilibrio dei naturali cicli chimici.
\begin{block}{Inquinamento atmosferico}
	\begin{center}
		\begin{figure}
			\includegraphics[scale=0.35]{Images/Immagini/Immagine2.jpg}
		\end{figure}
	\end{center}	
\end{block}
\end{frame}


\begin{frame}
\frametitle{Inquinamento atmosferico}
\textbf{Le sostanze inquinanti possono essere classificate in diversi modi:}
\begin{center}
	\begin{itemize}
		\item Secondo la loro origine.
		\begin{itemize}
			\item Naturale (incendi boschivi, eruzioni vulcaniche, emissioni $VOC_s$ dalla vegetazione ecc...).
			\item Antropica (attivit� industriali, traffico veicolare, riscaldamento civile e industriale ecc...).
		\end{itemize}
		\item Secondo la classe chimica di appartenenza.
		\begin{itemize}
			\item Sostanze organiche volatili (questa classe � numerosa e pu� essere suddivisa ulteriormente...). 
			\item Metalli pesanti.
			\item Ossidi inorganici.
		\end{itemize}
		\item Secondo il loro comportamento/evoluzione in atmosfera.
		\begin{itemize}
			\item Primari (emessi direttamente dalle sorgenti senza subire trasformazioni chimiche in atmosfera).
			\item Secondari (originano da trasformazioni chimiche in atmosfera con la fondamentale partecipazione di precursori).
		\end{itemize}
	\end{itemize}
\end{center}
\end{frame}


\begin{frame}	
	\frametitle{L'inquinamento atmosferico}
	\framesubtitle{Inquinamento tipico primaverile - estivo}
	Nell'inquinamento tipico primaverile estivo la fa da padrone l'ozono troposferico:
	\vspace{5mm}
	\begin{center}
	\begin{itemize}
		\item L'ozono � un inquinante di tipo secondario.
		\item Per la sua formazione necessita di precursori quali: $VOC_s$, $NO_x$ e radiazione solare.
		\item E' un forte ossidante tossico per la salute e dannoso per i materiali plastici.
	\end{itemize}
	\end{center}
\end{frame}



\begin{frame}	
	\frametitle{L'inquinamento atmosferico}
	\framesubtitle{Inquinamento tipico primaverile - estivo}
	La formazione fotochimica dell'ozono troposferico � influenzata pesantemente dalla presenza 
	di $NO_x$ e $VOC_s$. Le reazioni chimiche principali sono le seguenti:
	\vspace{10mm}
	\begin{columns}[c]  
		\column{2.5in}
		$NO_{2} + \alert{h\nu} (\lambda \leq 420 nm) \rightarrow NO + O(^{3}P)^{.}$\\
		\vspace{2mm}
		$O(^{3}P)^{.} + O_{2} + M \rightarrow O_{3} + M$\\
		\vspace{2mm}
		$NO + O_{3} \rightarrow NO_2 + O_{2}$\\
		\vspace{5mm}
		$HO_{2}^{.} + NO \rightarrow OH^{.} + NO_{2}$\\
		\vspace{2mm}
		$RO_{2}^{.} + NO \rightarrow RO^{.} + NO_{2}$\\
		\column{2in}
		\setbeamercolor{postut}{fg=black,bg=cyan}
		\begin{beamerboxesrounded}[upper=postit ,lower=postut ,shadow=true]{}
			La presenza di $VOC_s$ rende il ciclo di fotolisi dell' $NO_2$ instabile con 
			formazione netta di ozono.
		\end{beamerboxesrounded}
		\vspace{5mm}
	\end{columns}
\end{frame}  

\begin{frame}
	\frametitle{L'inquinamento atmosferico}
	\framesubtitle{Inquinamento tipico autunno - invernale}
	Nell'inquinamento tipico autunno - invernale i principali inquinanti incriminati sono gli 
	aerosol atmosferici e il biossido di azoto ($NO_2$). Le caratteristiche principali degli aerosol sono le seguenti:
	\vspace{10mm}
	\begin{itemize}
		\item Sono inquinanti sia di origine primaria che secondaria.
		\item Sono eterogenei nella loro composizione e forma/grandezza.
		\item La loro tossicit� varia a seconda della composizione chimico-fisica.
		\item Sono classificabili/campionabili secondo il loro diametro aerodinamico.
	\end{itemize}
\end{frame}

\begin{frame}
	\frametitle{L'inquinamento atmosferico}
	\framesubtitle{Inquinamento tipico autunno - invernale}
	\textbf{Il Particolato Atmosferico PM10 e PM2.5}
	\vspace{10mm}
	\begin{columns}[c]  
		\column{2in}
		\framebox{\includegraphics[width=2in]{Images/Immagini/Immagine8.jpg}} 
		\column{2in}
		\framebox{\includegraphics[width=1.5in]{Images/Immagini/Immagine9.jpg}} 
	\end{columns}
\end{frame}


\begin{frame}
	\frametitle{L'inquinamento atmosferico}
	\framesubtitle{Inquinamento tipico autunno - invernale}
	\textbf{Il Particolato Atmosferico (PM) - Uno sguardo al microscopio:}
	\vspace{10mm}
	\begin{columns}[c]  
		\column{1.5in}
		\framebox{\includegraphics[width=1.5in]{Images/Immagini/polvere1.jpg}} 
		\begin{center}
			\caption{PM10}
		\end{center}
		\column{1.5in}
		\framebox{\includegraphics[width=1.5in]{Images/Immagini/polvere2.jpg}}
		\begin{center}
			\caption{PM2.5}
		\end{center}
		\column{1.5in}
		\framebox{\includegraphics[width=1.5in]{Images/Immagini/polvere3.jpg}} 
		\begin{center}
			\caption{PM1}
		\end{center}
	\end{columns}
\end{frame}


\begin{frame}
	\frametitle{L'inquinamento atmosferico}
	\framesubtitle{Inquinamento tipico autunno - invernale}
	\textbf{Il Particolato Atmosferico (PM) - Uno sguardo al microscopio rispetto ad alcune sorgenti:}
	\vspace{10mm}
	\begin{columns}[c]  
		\column{1.5in}
		\framebox{\includegraphics[width=1.5in]{Images/Immagini/polvere4.jpg}}
		\begin{center}
			\caption{metano}
		\end{center}	
		\column{1.5in}
		\framebox{\includegraphics[width=1.5in]{Images/Immagini/polvere5.jpg}}
		\begin{center}
			\caption{legna}
		\end{center}
		\column{1.5in}
		\framebox{\includegraphics[width=1.5in]{Images/Immagini/polvere6.jpg}} 
		\begin{center}
			\caption{gasolio}
		\end{center}
	\end{columns}
\end{frame}

\begin{frame}
	
	\frametitle{Monitoraggio dell'inquinamento atmosferico}
	\small{Il monitoraggio degli inquinanti presenti in atmosfera si effettua utilizzando dalle pi\'u classiche 
	alle pi\'u moderne tecniche di chimica analitica.\\ ARPA Piemonte gestisce, dal punto di vista tecnico scientifico, 
	numerose stazioni di rilevamento degli inquinanti che contengono al loro interno gli strumenti atti alla quantificazione di ogni
	inquinante.}
	\vspace{3mm}
	\begin{columns}[c]  
		\column{1.5in}
		\framebox{\includegraphics[width=1.3in]{Images/Immagini/MMBi2.jpg}} 
		\column{1.5in}
		\framebox{\includegraphics[width=1.3in]{Images/Immagini/MMBi.jpg}} 
	\end{columns}
\end{frame}

\begin{frame}	
	\frametitle{Monitoraggio dell'inquinamento atmosferico}
	Arpa Piemonte dispone di 58 stazioni fisse di cui 4 private gestite da ARPA e 5 laboratori mobili.
	\vspace{5mm}
	\begin{columns}[c]  
		\column{1.5in}
		\framebox{\includegraphics[width=1.5in]{Images/Immagini/sistemarilevamento.jpg}} 
		\column{1.5in}
		\framebox{\includegraphics[width=1.3in]{Images/Immagini/STABi1.jpg}} 
	\end{columns}
\end{frame}

\begin{frame}
	\frametitle{Monitoraggio dell'inquinamento atmosferico}
	\textbf{I principali inquinanti chimici monitorati sono i seguenti:}
	\vspace{5mm}
	\begin{itemize}
		\item Monossido di carbonio ($CO$).
		\item Ossidi di azoto ($NO$ - \alert{$NO_2$}).
		\item Biossido di zolfo ($SO_2$).
		\item Ozono (\alert{$O_3$}).
		\item Materiale particolato \alert{$PM10$} e \alert{$PM2.5$}.
		\begin{itemize}
			\item Metalli pesanti ($As$, $Cd$, $Ni$, $Pb$).
			\item Idrocarburi policiclici aromatici ($Benzo[a]pirene$).
		\end{itemize}
		\item Idrocarburi aromatici ($BTX$).
	\end{itemize} 
\end{frame}

\begin{frame}
	\frametitle{Monitoraggio dell'inquinamento atmosferico}
	\begin{center}
		\textbf{Monossido di carbonio - $CO$}
	\end{center}
	\vspace{1mm}
	Gas tossico che si forma durante la combustione incompleta dei combustibili usati 
	negli autoveicoli e in generale in qualsiasi processo combustivo in difetto di ossigeno.\\
	\vspace{5mm}
	Il valore limite � di $10$ $mg/m^{3}$ come media massima giornaliera su 8 ore calcolata ogni ora sulla base delle otto
	ore precedenti.\\
	\vspace{1mm}
	Il suo monitoraggio viene effettuato utilizzando la tecnica di assorbimento IR non dispersivo.
	Una sorgente di radiazione IR policromatica viene fatta passare attraverso il campione e una cella di riferimento
	contenente un gas che non assorbe nell'IR. Un rivelatore a valle misura la differenza di riscaldamento che producono.
\end{frame}

\begin{frame}
	\frametitle{Monitoraggio dell'inquinamento atmosferico}
	\textbf{Schema strumentale per il monitoraggio del $CO$}
	\begin{center}
		\begin{columns}[c]  
			\column{2in}
			\framebox{\includegraphics[width=2in]{Images/Immagini/SchemaCO.jpg}}  
		\end{columns}
	\end{center}
\end{frame}

\begin{frame}
	\frametitle{Monitoraggio dell'inquinamento atmosferico}
	\begin{center}
		\textbf{Biossido di zolfo - $SO_2$}
	\end{center}
	
	\small{Si forma per combustione di materiali contenenti zolfo, quali i combustibili fossili come il 
	carbone, nafta e gasolio. Nell'aria si combina facilmente con l'acqua atmosferica dando inizialmente 
	acido solforoso e  successivamente, per ossidazione lenta, acido solforico.\\
	
	I valori limite dell' $SO_2$ sono diversi:
	\begin{itemize}
		\item $350$ $\mu g/m^{3}$ come valore limite orario da non superare pi\'u di 18 volte l'anno.
		\item $125$ $\mu g/m^{3}$ come valore limite giornaliero da non superare pi\'u di 3 volte l'anno.
		\item $500$ $\mu g/m^{3}$ soglia di allarme - media oraria per tre ore consecutive.
	\end{itemize}
	
	Il suo monitoraggio viene effettuato utilizzando la tecnica fluorimetrica, sfruttando 
	la caratteristica dell' $SO_2$ di emettere, se irradiata con una radiazione monocromatica opportuna, una radiazione a 
	lunghezza d'onda diversa da quella di incidenza.}
\end{frame}

\begin{frame}
	\frametitle{Monitoraggio dell'inquinamento atmosferico}
	\textbf{Schema strumentale per il monitoraggio del $SO_2$}
	\begin{center}
		\begin{columns}[c]  
			\column{3in}
			\framebox{\includegraphics[width=3in]{Images/Immagini/StrumSO2.jpg}}  
		\end{columns}
	\end{center}
\end{frame}

\begin{frame}
	\frametitle{Monitoraggio dell'inquinamento atmosferico}
	\begin{center}
		\textbf{Idrocarburi aromatici - $BTX$}
	\end{center}
	\vspace{1mm}
	Gli Idrocarburi aromatici sono una classe chimica di composti caratterizzati dalla presenza 
	di un anello benzenico.\\
	La loro concentrazione in atmosfera � direttamente correlabile al traffico veicolare: infatti il benzene �
	un componente importante delle benzine.\\
	\vspace{2mm}
	Il valore limite \'e di $5$ $\mu g/m^{3}$ come media annuale.\\
	\vspace{2mm}
	Il suo monitoraggio viene effettuato utilizzando la tecnica gascromatografica con 
	rivelatore PID a fotoionizzazione.
\end{frame}

\begin{frame}
	\frametitle{Monitoraggio dell'inquinamento atmosferico}
	\textbf{Schema strumentale per il monitoraggio degli idrocarburi $BTX$}
	\vspace{2mm}
	\begin{columns}[c]  
		\column{2in}
		\framebox{\includegraphics[width=2.0in]{Images/Immagini/schemagc.jpeg}}  
		\column{2in}
		\framebox{\includegraphics[width=2.0in]{Images/Immagini/PID.jpeg}}  
	\end{columns}
\end{frame}

\begin{frame}
	\frametitle{Monitoraggio dell'inquinamento atmosferico}
	\textbf{Strumenti e colonna per il monitoraggio degli $BTX$}
	\vspace{5mm}
	\begin{columns}[c]  
		\column{1.5in}
		\framebox{\includegraphics[width=1.5in]{Images/Immagini/gc1.jpeg}}  
		\column{1.5in}
		\framebox{\includegraphics[width=1.5in]{Images/Immagini/colonnagc.jpeg}}
		\column{1.5in}
		\framebox{\includegraphics[width=1.5in]{Images/Immagini/GCgramma.jpg}}
	\end{columns}
\end{frame}

\begin{frame}
	\frametitle{Monitoraggio dell'inquinamento atmosferico}
	\textbf{Alcuni dati sugli idrocarburi aromatici - $BTX$}
	\vspace{2mm}
	\begin{columns}[c]  
		\column{2in}
		\framebox{\includegraphics[width=1.9in]{Images/Immagini/SerieTempB2010.jpeg}}  
		\column{2in}
		\framebox{\includegraphics[width=1.9in]{Images/Immagini/MediaBBar.jpeg}}
	\end{columns}
\end{frame}

\begin{frame}
	\frametitle{Monitoraggio dell'inquinamento atmosferico}
	\textbf{Alcuni dati sugli idrocarburi aromatici - $BTX$}
	\vspace{2mm}
	\begin{columns}[c]  
		\column{2in}
		\framebox{\includegraphics[width=1.9in]{Images/Immagini/GiornoMedioBiLamB2010.jpeg}}  
		\column{2in}
		\framebox{\includegraphics[width=1.9in]{Images/Immagini/GiornoMedioCossB2010.jpeg}}
	\end{columns}
\end{frame}

\begin{frame}
	\frametitle{Monitoraggio dell'inquinamento atmosferico}
	\textbf{Alcuni dati sugli idrocarburi aromatici - $BTX$}
	\vspace{2mm}
	\begin{center}
		\begin{columns}[c]  
			\column{3in}
			\framebox{\includegraphics[width=3in]{Images/Immagini/ValBenz.jpg}}  
		\end{columns}
	\end{center}
\end{frame}

\begin{frame}
	\frametitle{Monitoraggio dell'inquinamento atmosferico}
	\begin{center}
		\textbf{Ozono - $O_3$}
	\end{center}
	\vspace{2mm}
	L'ozono � un gas pungente che non ha sorgenti dirette significative, ma si produce 
	all'interno di un ciclo di reazioni fotochimiche.\\
	\vspace{2mm}
	I valori limite sono i seguenti:\\
	\begin{itemize}
		\item $120$ $\mu g/m^{3}$ come valore bersaglio da non superare pi� di 25 volte l'anno 
		(max media mobile su 8 ore).
		\item $180$ $\mu g/m^{3}$ come soglia di informazione (media oraria).
		\item $240$ $\mu g/m^{3}$ come soglia di allarme per tre ore consecutive (media oraria).
	\end{itemize}
	\vspace{2mm}
	Il suo monitoraggio viene effettuato utilizzando la tecnica spettrofotometrica.
\end{frame}

\begin{frame}
	\frametitle{Monitoraggio dell'inquinamento atmosferico}
	\textbf{Schema strumentale per il monitoraggio dell'Ozono - $O_3$}
	\vspace{10mm}
	\begin{center}
		\begin{columns}[c]  
			\column{3in}
			\framebox{\includegraphics[width=3.0in]{Images/Immagini/ozonoanalisi.jpg}}  
		\end{columns}
	\end{center}
\end{frame}

\begin{frame}
	\frametitle{Monitoraggio dell'inquinamento atmosferico}
	\textbf{Alcuni dati sull'Ozono - $O_3$}
	\vspace{5mm}
	\begin{columns}[c]  
		\column{2in}
		\framebox{\includegraphics[width=1.9in]{Images/Immagini/SerieTempO32010.jpeg}}  
		\column{2in}
		\framebox{\includegraphics[width=1.9in]{Images/Immagini/NSupO3Bar.jpeg}}
	\end{columns}
\end{frame}

\begin{frame}
	\frametitle{Monitoraggio dell'inquinamento atmosferico}
	\textbf{Alcuni dati sull'Ozono - $O_3$}
	\vspace{5mm}
	\begin{columns}[c]  
		\column{2in}
		\framebox{\includegraphics[width=1.9in]{Images/Immagini/GiornoMedioBiStuO32010.jpeg}}  
		\column{2in}
		\framebox{\includegraphics[width=1.9in]{Images/Immagini/GiornoMedioCossO32010.jpeg}}
	\end{columns}
\end{frame}

\begin{frame}
	\frametitle{Monitoraggio dell'inquinamento atmosferico}
	\textbf{Alcuni dati sull'Ozono - $O_3$}
	\vspace{5mm}
	\begin{center}
		\begin{columns}[c]  
			\column{2.5in}
			\framebox{\includegraphics[width=2.4in]{Images/Immagini/Labplot1.jpeg}}  
			\column{2.5in}
			\framebox{\includegraphics[width=2.4in]{Images/Immagini/Labplot2.jpeg}}  
		\end{columns}
	\end{center}
\end{frame}


\begin{frame}
	\frametitle{Monitoraggio dell'inquinamento atmosferico}
	\begin{center}
		\textbf{Biossidi di azoto - $NO_x$}
	\end{center}
	\vspace{5}
	Gli $NO_x$ comprendono principalmente gli $NO$ e gli $NO_2$ sono composti che si formano nei processi di combustione ad alta
	temperatura e sono i principali attori nella formazione del particolato e dello smog fotochimico.\\
	\vspace{5}
	I valori limite sono:
	\begin{itemize}
		\item $200$ $\mu g/m^{3}$ come valore limite orario da non superare 
		pi\'u di 18 volte l'anno.
		\item $40$ $\mu g/m^{3}$ come valore limite annuale.
		\item $400$ $\mu g/m^{3}$ soglia di allarme - media oraria per tre ore consecutive.
	\end{itemize}
	\vspace{5}
	Il loro monitoraggio viene effettuato utilizzando la tecnica per chemiluminescenza.
\end{frame}

\begin{frame}
	\frametitle{Monitoraggio dell'inquinamento atmosferico}
	\textbf{Schema strumentale per il monitoraggio degli $NO_x$}
	\vspace{15}
	\begin{columns}[c]  
		\column{2in}
		\framebox{\includegraphics[width=2.1in]{Images/Immagini/StrumNOx.jpg}}
		\column{2in}
		$NO + O_3 \longrightarrow NO_2^\ast + O_2$
		\vspace{5}
		$NO_2^\ast \longrightarrow NO_2 + h\nu$
		\vspace{5}
		$2NO_2 + Mo \longrightarrow NO + MoO_3 $
	\end{columns}
\end{frame}

\begin{frame}
	\frametitle{Monitoraggio dell'inquinamento atmosferico}
	\textbf{Alcuni dati sugli ossidi di azoto - $NO_x$}
	\vspace{5}
	\begin{columns}[c]  
		\column{2in}
		\framebox{\includegraphics[width=1.9in]{Images/Immagini/SerieTempNO2010.jpeg}}  
		\column{2in}
		\framebox{\includegraphics[width=1.9in]{Images/Immagini/SerieTempNO22010.jpeg}}
	\end{columns}
\end{frame}


\begin{frame}
	\frametitle{Monitoraggio dell'inquinamento atmosferico}
	\textbf{Alcuni dati sugli ossidi di azoto - $NO_x$}
	\vspace{5}
	\begin{center}
		\begin{columns}[c]  
			\column{3in}
			\framebox{\includegraphics[width=3in]{Images/Immagini/ValNOx.jpg}}  
		\end{columns}
	\end{center}
\end{frame}


\begin{frame}
	\frametitle{Monitoraggio dell'inquinamento atmosferico}
	\begin{center}
		\textbf{Particolato atmosferico - $PM$}
	\end{center}
	\small{Il particolato atmosferico � una sospensione complessa di particelle solide, liquide o miste in equilibrio
	con la fase gassosa. Le sorgenti sono sia primarie che secondarie e le principali sono il traffico veicolare
	, gli impianti di riscaldamento sia civili che industriali. Il particolato con diametro aerodinamico inferiore 
	a $10$ $\mu m$ \'e di particolare interesse tossicologico.
	\vspace{1}
	I valori limite per il PM10 sono:
	\begin{itemize}
		\item $50$ $\mu g/m^{3}$ come valore limite giornaliero (media giornaliera).
		\item $40$  $\mu g/m^{3}$ come valore limite annuale.
		\item $35$ numero massimo di superamenti del limite giornaliero in un anno.
	\end{itemize}}
\end{frame}


\begin{frame}
	\frametitle{Monitoraggio dell'inquinamento atmosferico}
		\begin{center}
		\textbf{Particolato atmosferico - $PM$}
	\end{center}
Il suo monitoraggio viene effettuato in diversi modi:
\begin{itemize}
	\item Tecniche in continuo.
	\begin{itemize}
		\item TEOM - microbilancia a smorzamento di oscillazione.
		\item Attenuazione raggi $\beta$.
		\item Scattering radiazione elettromagnetica.
	\end{itemize}
	\item Tecnica discontinua - gravimetrica (metodo ufficiale).
\end{itemize}

\end{frame}


\begin{frame}
	\frametitle{Monitoraggio dell'inquinamento atmosferico}
	\textbf{Strumenti per il monitoraggio in continuo del particolato atmosferico - $PM$}
	\vspace{10}
	\begin{columns}[c]  
		\column{1.5in}
		\framebox{\includegraphics[width=1.3in]{Images/Immagini/TestaPMamericana.jpg}}  
		\column{1.5in}
		\framebox{\includegraphics[width=1.1in]{Images/Immagini/Teom.jpg}}  
	\end{columns}
\end{frame}

\begin{frame}
	\frametitle{Monitoraggio dell'inquinamento atmosferico}
	\textbf{Strumenti per il monitoraggio gravimetrico del particolato atmosferico - $PM$}
	\vspace{10}
	\begin{columns}[c]  
		\column{1.5in}
		\framebox{\includegraphics[width=1.5in]{Images/Immagini/Skypost.jpg}}  
		\column{1.5in}
		\framebox{\includegraphics[width=1.5in]{Images/Immagini/SchemaTestaPM.jpg}}
		\column{1.5in}
		\framebox{\includegraphics[width=1.5in]{Images/Immagini/flussoPesoPM.jpg}}
	\end{columns}
\end{frame}

\begin{frame}
	\frametitle{Monitoraggio dell'inquinamento atmosferico}
	\textbf{La speciazione del particolato atmosferico - $PM$}\\
	\vspace{10}
	Successivamente al campionamento e alla determinazione gravimetrica del PM, i campioni vengono trattati e analizzati per determinare
	il loro contenuto in metalli, IPA e diversi altri componenti chimici.\\
	\vspace{10}
	\begin{columns}[c]  
		\column{2.5in}
		\framebox{\includegraphics[width=2.5in]{Images/Immagini/filtroaliquote.jpeg}}  
	\end{columns}
\end{frame}

\begin{frame}
	\frametitle{Monitoraggio dell'inquinamento atmosferico}
	\textbf{La speciazione del particolato atmosferico - $PM$}\\
	\vspace{10}
	La speciazione chimica del particolato atmosferico si effettua con diverse techiche di chimica analitica 
	previo adeguato trattamento del campione:\\
	\vspace{10}
	\begin{itemize}
		\item Metalli pesanti (ICP/MS, ICP/AES).
		\item Idrocarburi policiclici aromatici (GC/MS, HPLC).
		\item Anioni e cationi (CI)
		\item Sostanze organiche (CG/MS, HPLC).
	\end{itemize}
\end{frame}

\begin{frame}
	\frametitle{Monitoraggio dell'inquinamento atmosferico}
	\textbf{La speciazione del particolato atmosferico $PM$ - Determinazione dei metalli}\\
	\vspace{10}
	Spettrometria di massa al plasma accoppiato induttivamente - ICP/MS
	\vspace{10}
	\begin{columns}[c]  
		\column{2.5in}
		\framebox{\includegraphics[width=2.5in]{Images/Immagini/icp-ms.jpg}}  
		\column{2in}
		\framebox{\includegraphics[width=2in]{Images/Immagini/ICPMS.jpeg}}
	\end{columns}
\end{frame}

\begin{frame}
	\frametitle{Monitoraggio dell'inquinamento atmosferico}
	\textbf{La speciazione del particolato atmosferico $PM$ - Determinazione delle Sostanze organiche}\\
	\vspace{10}
	Gascromatografia accoppiata alla spettrometria di massa - GC/MS
	\vspace{10}
	\begin{columns}[c]  
		\column{1.5in}
		\framebox{\includegraphics[width=1.3in]{Images/Immagini/AnalFrazOrganica.jpg}}  
		\column{1.5in}
		\framebox{\includegraphics[width=1.5in]{Images/Immagini/gc1.jpeg}}
	\end{columns}
\end{frame}

\begin{frame}
	\frametitle{Monitoraggio dell'inquinamento atmosferico}
	\textbf{La speciazione del particolato atmosferico $PM$ - Determinazione dei cationi e anioni}\\
	\vspace{10}
	Cromatografia ionica - CI
	\vspace{5}
	\begin{columns}[c]  
		\column{2in}
		\framebox{\includegraphics[width=2in]{Images/Immagini/schemaCI.jpeg}}  
		\column{2in}
		\framebox{\includegraphics[width=2in]{Images/Immagini/CI.jpg}}
	\end{columns}
\end{frame}

\begin{frame}
	\frametitle{Monitoraggio dell'inquinamento atmosferico}
	\textbf{Alcuni dati sul particolato atmosferico - $PM$}
	\vspace{5}
	\begin{columns}[c]  
		\column{2in}
		\framebox{\includegraphics[width=1.9in]{Images/Immagini/MediaPM10Bar.jpeg}}  
		\column{2in}
		\framebox{\includegraphics[width=1.9in]{Images/Immagini/NSupPM10Bar.jpeg}}
	\end{columns}
\end{frame}

\begin{frame}
	\frametitle{Monitoraggio dell'inquinamento atmosferico}
	\textbf{Alcuni dati sul particolato atmosferico - $PM$}
	\vspace{5}
	\begin{columns}[c]  
		\column{2in}
		\framebox{\includegraphics[width=1.9in]{Images/Immagini/GiornoMedioBiStuPM10Teom2010.jpeg}}  
		\column{2in}
		\framebox{\includegraphics[width=1.9in]{Images/Immagini/GiornoMedioCossPM10Teom2010.jpeg}}
	\end{columns}
\end{frame}

\begin{frame}
	\frametitle{Monitoraggio dell'inquinamento atmosferico}
	\textbf{Alcuni dati sul particolato atmosferico - $PM$}
	\vspace{5}
	\begin{center}
		\begin{columns}[c]  
			\column{3in}
			\framebox{\includegraphics[width=3in]{Images/Immagini/ValPM10Teom.jpg}}  
		\end{columns}
	\end{center}
\end{frame}


\begin{frame}
	\frametitle{Monitoraggio dell'inquinamento atmosferico}
	\textbf{Alcuni dati sul particolato atmosferico - $PM$}
	\vspace{5}
	\begin{center}
		\begin{columns}[c]  
			\column{2in}
			\framebox{\includegraphics[width=2in]{Images/Immagini/BiPM102008.jpeg}}  
			\column{1.5in}
			\framebox{\includegraphics[width=1.9in]{Images/Immagini/sat.jpeg}}
		\end{columns}
	\end{center}
\end{frame}

\begin{frame}
	\frametitle{Monitoraggio dell'inquinamento atmosferico}
	\textbf{Alcuni dati sul particolato atmosferico - $PM$}
	\vspace{5}
	\begin{center}
		\begin{columns}[c]  
			\column{2in}
			\framebox{\includegraphics[width=2in]{Images/Immagini/meteo11.jpg}}  
			\column{1.9in}
			\framebox{\includegraphics[width=1.9in]{Images/Immagini/velvento.jpeg}}
		\end{columns}
	\end{center}
\end{frame}

\begin{frame}
	\frametitle{Monitoraggio dell'inquinamento atmosferico}
	\textbf{Alcuni dati sul particolato atmosferico - $PM$}
	\vspace{5}
	\begin{center}
		\begin{columns}[c]  
			\column{2in}
			\framebox{\includegraphics[width=2in]{Images/Immagini/meteo6.jpg}}  
			\column{2in}
			\framebox{\includegraphics[width=2in]{Images/Immagini/meteo1.jpg}}
		\end{columns}
	\end{center}
\end{frame}

%%%%%%%%%%%%%%%%%%%%%%%%%%%%%%%%%%%%%%%%%%%%%%%%%%%%%%%%%%%%%%%%%%%%%%%%%%%%%%%%%%%%%%%%%%%%%%%%%%%%%%%%%%%%%%%%%%%%%%%%
%
%  Slides - Tossicologia Inquinanti Atmosferici
%
%%%%%%%%%%%%%%%%%%%%%%%%%%%%%%%%%%%%%%%%%%%%%%%%%%%%%%%%%%%%%%%%%%%%%%%%%%%%%%%%%%%%%%%%%%%%%%%%%%%%%%%%%%%%%%%%%%%%%%%%

\begin{frame}
	\frametitle{Effetti dell'inquinamento sulla salute umana}
	\small{Il monitoraggio degli inquinanti atmosferici ha come fine principale la salvaguardia della salute umana.
	Alla fine dell'ottocento la sensibilit\'a verso l'inquinamento atmosferico ha iniziato ad aumentare significativamente a 
	seguito di vari episodi eclatanti di morte causata da aria insalubre:
	\begin{itemize}
		\item 1873/1963 - Londra
		\begin{itemize}
			\item 1873 - 500 persone morte.
			\item 1880 - 1000 persone morte.
			\item 1892 - 1000 persone morte.
			\item 1948 al 1962 ci furono diversi episodi ma il pi\'u disastroso f\'u nel dicembre del 1952 (4000 mila morti in 5 giorni)
			($SO_2$).
		\end{itemize}
		\item 1984 - Disastro di Bhopal pi\'u di 2000 morti e 300000 feriti ($CH_3CN$).
		\item 1948 - Citt\'a di Donora USA 20 morti in 14 ore ($F, S, CO,$ metalli pesanti).
		\item 1930 - Valle di Meuse diverse migliaia di casi di attacchi polmonari acuti e 60 morti ($SO_2$).
		\item 1950 - Messico/Poza Rica 320 persone ospedalizzate e 22 morti\\ in 3 ore ($H_2S$).
	\end{itemize}}   
\end{frame}

\begin{frame}
	\transsplitverticalin
	\frametitle{Effetti dell'inquinamento sulla salute umana}
	\textbf{Londra - Dicembre 1952}
	\vspace{5}
	\begin{center}
		\begin{columns}[c]  
			\column{2in}
			\framebox{\includegraphics[width=1.5in]{Images/Immagini/london-smog-1.jpg}}  
			\column{2in}
			\framebox{\includegraphics[width=1.5in]{Images/Immagini/smog-data.jpg}}
		\end{columns}
	\end{center}
\end{frame}

\begin{frame}
	\frametitle{Effetti dell'inquinamento sulla salute umana}
	\textbf{Cenni di tossicologia degli inquinanti} 
	\begin{itemize}
		\item Monossido di carbonio - $CO$
		\begin{itemize}
			\item Il $CO$ si lega all'emoglobina, con una affinit\'a 200 volte superiore all'ossigeno, formando la carbossiemoglobina che
			impedisce il traspoto di ossigeno ai vari distretti corporei.
		\end{itemize}
		\item Biossido di zolfo - $SO_2$
		\begin{itemize}
			\item L'$SO_2$ \'e un gas molto irritante; causa, anche a basse concentrazioni, irritazione alle vie 
			respiratorie. Esposizioni croniche comportano faringiti e affaticamento/disturbi 
			all'apparato respiratorio.
		\end{itemize}
		\item Benzene
		\begin{itemize}
			\item Il benzene \'e una sostanza classificata cancerogena per l'uomo.
		\end{itemize}
		\item Biossido di azoto $NO_2$.
		\begin{itemize}Pesticidi
			\item L'$NO_2$ \'e un gas tossico e irritante per le mucose ed \'e responsabile di patologie a 
			carico del sistema respiratorio.
		\end{itemize}
	\end{itemize}   
\end{frame}

\begin{frame}
	\frametitle{Effetti dell'inquinamento sulla salute umana}
	\textbf{Cenni di tossicologia dell'ozono}
	\vspace{2mm}
	\begin{center}
		\begin{columns}[c]  
			\column{2.5in}
			\framebox{\includegraphics[width=2.45in]{Images/Immagini/Immagine4.jpg}}  
			\column{2in}
			\setbeamercolor{postut}{fg=black,bg=cyan}
			\begin{beamerboxesrounded}[upper=postit ,lower=postut ,shadow=true]{}
				Concentrazioni relativamente basse di ozono provocano effetti quali irritazione alla gola, alle vie
				respiratorie e bruciore agli occhi; concentrazioni pi� elevate alterano le funzioni respiratorie.
			\end{beamerboxesrounded}
		\end{columns}
	\end{center}
\end{frame}

\begin{frame}
	\frametitle{Effetti dell'inquinamento sulla salute umana}
	\textbf{Cenni di tossicologia del particolato atmosferico}
	\vspace{5}
	\begin{center}
		\begin{columns}[c]  
			\column{2.5in}
			\framebox{\includegraphics[width=2.5in]{Images/Immagini/DeadPM1.jpeg}}  
		\end{columns}
	\end{center}
\end{frame}

\begin{frame}
	\frametitle{Effetti dell'inquinamento sulla salute umana}
	\textbf{Cenni di tossicologia del particolato atmosferico}
	\vspace{5}
	\begin{center}
		\begin{columns}[c]  
			\column{2in}
			\framebox{\includegraphics[width=2in]{Images/Immagini/polmoni.jpg}}
			\column{2in}
			\framebox{\includegraphics[width=1.9in]{Images/Immagini/TossicologiaBaP.jpg}}
		\end{columns}
	\end{center}
\end{frame}

\begin{frame}
	\frametitle{Effetti dell'inquinamento sulla salute umana}
	\textbf{Cenni di tossicologia del particolato atmosferico}
	\vspace{5}
	\begin{center}
		\begin{columns}[c]  
			\column{2in}
			\framebox{\includegraphics[width=1.7in]{Images/Immagini/CardioPM.jpeg}}  
		\end{columns}
	\end{center}
\end{frame}

%%%%%%%%%%%%%%%%%%%%%%%%%%%%%%%%%%%%%%%%%%%%%%%%%%%%%%%%%%%%%%%%%%%%%%%%%%%%%%%%%%%%%%%%%%%%%%%%%%%%%%%%%%%%%%%%%%%%%%%%
%
%  Slides - Pesticidi
%
%%%%%%%%%%%%%%%%%%%%%%%%%%%%%%%%%%%%%%%%%%%%%%%%%%%%%%%%%%%%%%%%%%%%%%%%%%%%%%%%%%%%%%%%%%%%%%%%%%%%%%%%%%%%%%%%%%%%%%%%


\begin{frame}
\frametitle{Cenni sull'inquinamento da fitofarmaci}
I \textbf{fitofarmaci} denominati anche \textbf{prodotti fitosanitari} o \textbf{agrofarmaci}  sono tutti quei prodotti, di sintesi o naturali, che vengono utilizzati per combattere le principali avversit� delle piante quali malattie infettive, fisiopatie, parassiti e fitofagi animali, piante infestanti.
\vspace{2mm}
\begin{center}
		\framebox{\includegraphics[width=2.5in]{Images/Immagini/pesticidi1.jpg}}  
\end{center}
\end{frame}




\begin{frame}
\frametitle{Classificazione dei fitofarmaci}

\end{frame}



%%%%%%%%%%%%%%%%%%%%%%%%%%%%%%%%%%%%%%%%%%%%%%%%%%%%%%%%%%%%%%%%%%%%%%%%%%%%%%%%%%%%%%%%%%%%%%%%%%%%%%%%%%%%%%%%%%%%%%%%
%
%  Slides - Conclusioni
%
%%%%%%%%%%%%%%%%%%%%%%%%%%%%%%%%%%%%%%%%%%%%%%%%%%%%%%%%%%%%%%%%%%%%%%%%%%%%%%%%%%%%%%%%%%%%%%%%%%%%%%%%%%%%%%%%%%%%%%%%

\begin{frame}
	\frametitle{Conclusioni}
	\begin{center}
		\textbf{La Biosfera � un sistema complesso e come tutti i sistemi complessi � molto delicata.\\ 
		\'E nostro compito preservarla per noi e per i nostri figli.\\
	    Non dimentichiamoci che i nostri comportamenti sono artefici del nostro destino e del destino della terra.}   
	\end{center}
	\begin{center}
		\begin{columns}[c]  
			\column{1.8in}
			\framebox{\includegraphics[width=1.7in]{Images/Immagini/carov.jpg}} 
		\end{columns}
	\end{center}
\end{frame}

\begin{frame}
	\frametitle{Bibliografia}
	\begin{itemize}
		\item Roger Atkinson - Atmospheric chemistry of $VOC_s$ and $NO_x$ - Atmospheric Environment 34 (2000) 2063-2101 
		\item H.R. Anderson - Air pollution and mortality: A history - Atmospheric Environment 43 (2009) 142-152
		\item Z. Boris - Air pollution and Cardiovascular Injury - Journal of American College of Cardiology 52-9 (2008)
		\item Pitts - Chemistry of the upper and lower atmosphere - Academic Press 2000
		\item Casarett \& Doll's - Tossicologia 
		\item D.lgs 13 agosto 2010 n. 155  
	\end{itemize}     
\end{frame}

\begin{frame}
	\frametitle{Fine}
	\begin{columns}[c]  
		\column{1in}
		\framebox{\includegraphics[width=1in]{Images/Immagini/Ubuntu.jpeg}} 
		\column{1in}
		\framebox{\includegraphics[width=1in]{Images/Immagini/Latex.jpeg}}
	\end{columns}
	\begin{center}
		\begin{LARGE}  
			\textbf{Grazie per l'attenzione!!!}
		\end{LARGE}
	\end{center} 
	\begin{columns}[c]  
		\column{1in}
		\framebox{\includegraphics[width=1in]{Images/Immagini/R.jpeg}}
		\column{1in}
		\framebox{\includegraphics[width=1in]{Images/Immagini/LABplot.jpg}}
	\end{columns}    
\end{frame}
\end{document}
