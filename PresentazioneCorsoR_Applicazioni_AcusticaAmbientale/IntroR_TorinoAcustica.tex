%~~~~~~~~~~~~~~~~~~~~~~~~~~~~~~~~~~~~~~~~~~~~~~~~~~~~~~~~~~~~~~~~~~~~~~~~~~~~~~~
%Presentazione: Corso: L'ambiente R - Uso in ambito Acustico -
%Luogo: Torino - Agenti Fisici - Acustica Ambientale
%Autore: Pasquale Scordino
%Data: 17 - 18 - 19 Ottobre 2018
%~~~~~~~~~~~~~~~~~~~~~~~~~~~~~~~~~~~~~~~~~~~~~~~~~~~~~~~~~~~~~~~~~~~~~~~~~~~~~~~
\documentclass{beamer}
\usetheme{CambridgeUS}
\usecolortheme{beaver}
\usefonttheme[onlymath]{serif}

%Settaggio colori e forma degli elenchi puntati
\setbeamercolor{itemize item}{fg=darkred!80!black}
\setbeamercolor{itemize subitem}{fg=orange}
\setbeamercolor{itemize subsubitem}{fg=cyan}

\setbeamertemplate{itemize item}[triangle]
\setbeamertemplate{itemize subitem}[circle]
\setbeamertemplate{itemize subsubitem}[triangle]

%Settaggio colore testo e sfondo del titolo del blocco
\setbeamercolor{block title}{fg=white,bg=darkred!80!black}
%~~~~~~~~~~~~~~~~~~~~~~~~~~~~~~~~~~~~~~~~~~~~~~~~~~~~~~~~~~~~~~~~~~~~~~~~~~~~~~~~
%Pacchetti usati
\usepackage[italian]{babel}
\usepackage[latin1]{inputenc}
\usepackage{times}
\usepackage[T1]{fontenc}
\usepackage{graphics}
\usepackage{subfigure}
\usepackage{multimedia}
\usepackage{pifont,xcolor} % http://ctan.org/pkg/{pifont,xcolor}
\usepackage{fancybox}
\usepackage{textpos}
\usepackage{amsmath}
\usepackage{xcolor}
\usepackage[document]{ragged2e} % per giustificare il testo
%~~~~~~~~~~~~~~~~~~~~~~~~~~~~~~~~~~~~~~~~~~~~~~~~~~~~~~~~~~~~~~~~~~~~~~~~~~~~~~~~~
%Pacchetto usato per scrivere codice R in Beamer e suo settaggio
\usepackage{listings}
\usepackage{color}
\definecolor{dkgreen}{rgb}{0,0.6,0}
\definecolor{gray}{rgb}{0.5,0.5,0.5}
\definecolor{mauve}{rgb}{0.58,0,0.82}
\lstset{ %
	language=R,                     % the language of the code
	basicstyle=\footnotesize,       % the size of the fonts that are used for the code
	numbers=left,                   % where to put the line-numbers
	numberstyle=\tiny\color{gray},  % the style that is used for the line-numbers
	stepnumber=1,                   % the step between two line-numbers. If it's 1, each line
	% will be numbered
	numbersep=5pt,                  % how far the line-numbers are from the code
	backgroundcolor=\color{white},  % choose the background color. You must add \usepackage{color}
	showspaces=false,               % show spaces adding particular underscores
	showstringspaces=false,         % underline spaces within strings
	showtabs=false,                 % show tabs within strings adding particular underscores
	frame=single,                   % adds a frame around the code
	rulecolor=\color{black},        % if not set, the frame-color may be changed on line-breaks within not-black text (e.g. commens (green here))
	tabsize=2,                      % sets default tabsize to 2 spaces
	captionpos=b,                   % sets the caption-position to bottom
	breaklines=true,                % sets automatic line breaking
	breakatwhitespace=false,        % sets if automatic breaks should only happen at whitespace
	title=\lstname,                 % show the filename of files included with \lstinputlisting;
	% also try caption instead of title
	keywordstyle=\color{blue},      % keyword style
	commentstyle=\color{dkgreen},   % comment style
	stringstyle=\color{mauve},      % string literal style
	escapeinside={\%*}{*)},         % if you want to add a comment within your code
	morekeywords={*,...}            % if you want to add more keywords to the set
}

\newsavebox{\mybox}

%~~~~~~~~~~~~~~~~~~~~~~~~~~~~~~~~~~~~~~~~~~~~~~~~~~~~~~~~~~~~~~~~~~~~~~~~~~~~~~~~~
%Prima slide intestazione 
\title[R - Applicazioni in Acustica Ambientale]{\textbf{L'ambiente} \textcolor{blue}{\textit{\textbf{R}}} \\ \textbf{Applicazioni in Acustica Ambientale}}

\author[P. Scordino\hspace{0.45cm}{p.scordino@arpa.piemonte.it}]{P. Scordino, S. Bande, C. Ronchi}
\institute{ARPA Piemonte}

\titlegraphic{

  \includegraphics[width=2cm,height=2cm,keepaspectratio]{Images/Arpacolori.jpg}%
  \hspace*{8.20cm}
  \includegraphics[width=2cm,height=2cm,keepaspectratio]{Images/logo_SNPA_COL.jpg}%

}
%~~~~~~~~~~~~~~~~~~~~~~~~~~~~~~~~~~~~~~~~~~~~~~~~~~~~~~~~~~~~~~~~~~~~~~~~~~~~~~~~~
%Inizio 
\begin{document}

\begin{frame}
\titlepage
\end{frame}

\begin{frame}
	\frametitle{Generalit� sull'analisi dei dati}
		\justify	
		Grazie \underline{all'evoluzione tecnologica}, che ha reso fruibile a basso costo l'hardware per la \textbf{memorizzazione} dei \textbf{dati} e l'hardware per \textbf{l'elaborazione} sempre pi� \textbf{veloce} degli stessi, oggi possiamo contare su una moltitudine di dati disponibili che ha reso necessario l'utilizzo di strumenti software potenti e flessibili per poter \textbf{estrarre informazioni utili in tempi ragionevoli.}
\end{frame}

\begin{frame}
\frametitle{Generalit� sull'analisi dei dati}
\begin{center}
	\begin{block}{Definizione di dati e informazione}
		\begin{itemize}
			\vspace{2mm}
			\item \textbf{Dato} dal latino \textit{datum}, dono. I dati sono rappresentazioni di eventi o fatti non interpretate, attraverso simboli e contenute su supporti.
			\vspace{3mm}
			\item \textbf{Informazione} dal latino \textit{informatio(-nis)} (nel significato di dare forma alla mente, disciplinare, istruire, insegnare). L'informazione � un insieme di dati, interpretati e comprensibili per il destinatario.
			\vspace{2mm}
		\end{itemize}	
	\end{block}
\end{center}
\end{frame}

\begin{frame}
\frametitle{Generalit� sull'analisi dei dati}
\begin{center}
	\begin{block}{Il flusso delle informazioni}
		\begin{figure}
			\includegraphics[scale=0.35]{Images/datascienceworkflow.jpeg}
		\end{figure}
	\end{block}
\end{center}
\end{frame}

\begin{frame}
\frametitle{L'ambiente \textcolor{blue}{\textit{\textbf{R}}} per l'analisi dei dati}
\small L'ambiente \textcolor{blue}{\textit{\textbf{R}}} per l'analisi dei dati � particolarmente utile in tutti i campi del sapere dove � necessario processare, analizzare e estrarre informazioni da grandi quantit� di dati.
	\begin{center}
		\begin{block}{Un p� di statistiche (-:}
	 		\begin{figure}
	 			\subfigure[]{
					\includegraphics[scale=0.15]{Images/WhatisRProg1.png}}
					\hspace{6mm}
				\subfigure[]{
					\includegraphics[scale=0.27]{Images/trend.png}}
			\end{figure}
		\end{block}
	\end{center}
\end{frame}

\begin{frame}
\frametitle{Uso di \textcolor{blue}{\textit{\textbf{R}}} in ambito acustico ambientale}
L'uso di \textcolor{blue}{\textit{\textbf{R}}} in acustica ambientale diventa indispensabile soprattutto nei monitoraggi acustici di \textbf{lungo periodo} e laddove c'� la necessit� di \textbf{confrontare} diversi dataset (meteo, idrometrici, ecc...) con i dati fonometrici:
\begin{itemize}
	\vspace{3mm}
	\item Reti di monitoraggio \textbf{rumore urbano.}  
	\item Reti di monitoraggio \textbf{rumore aereoportuale.}	
	\item Campagne di misurazione \textbf{rumore da traffico stradale e ferroviario.}	
	\item \textbf{Casi particolari di esposto} dove � necessaria una misura prolungata e in continuo per caratterizzare il disturbo della sorgente in esame.
\end{itemize}	
	
\end{frame}

\begin{frame}
\frametitle{Uso di \textcolor{blue}{\textit{\textbf{R}}} in ambito acustico ambientale}
\begin{block}{Esempi elaborazioni grafiche}
	\begin{figure}
		\subfigure[]{
			\includegraphics[scale=0.19]{Images/Rplot_Cicli_Manna.pdf}}
		\subfigure[]{
			\includegraphics[scale=0.19]{Images/Rplot_SpettroLLeq_Manna.pdf}}
	\end{figure}
\end{block}
\end{frame}

\begin{frame}
\frametitle{Uso di \textcolor{blue}{\textit{\textbf{R}}} in ambito acustico ambientale}
\begin{block}{Esempi elaborazioni grafiche}
	\begin{figure}
		\subfigure[]{
			\includegraphics[scale=0.25]{Images/Rbarplot_Manna_AnalisiSpettrale.pdf}}
	\end{figure}
\end{block}
\end{frame}

\begin{frame}
\frametitle{Uso di \textcolor{blue}{\textit{\textbf{R}}} in ambito acustico ambientale}
\begin{block}{Esempi elaborazioni grafiche}
	\begin{figure}
		\subfigure[]{
			\includegraphics[scale=0.19]{Images/Rplot_Portata_8_9_maggio_2017_ver2.pdf}}
		\subfigure[]{
			\includegraphics[scale=0.19]{Images/RplotTrend.pdf}}
	\end{figure}
\end{block}
\end{frame}

\begin{frame}
\frametitle{Uso di \textcolor{blue}{\textit{\textbf{R}}} in ambito acustico ambientale}
\begin{block}{Esempi elaborazioni grafiche}
	\begin{figure}
		\subfigure[]{
			\includegraphics[scale=0.19]{Images/Rplot22122016_Ambientale.pdf}}
		\subfigure[]{
			\includegraphics[scale=0.19]{Images/Rplot04012017_Residuo.pdf}}
	\end{figure}
\end{block}
\end{frame}

\begin{frame}
	\frametitle{Caso studio: Esposto}
	L'esposto che esamineremo ricade nei casi complessi in quanto l'esponente lamenta un disturbo rumoroso proveniente da una Azienda che dista dall'abitazione dell'esponente circa 300 metri ed inoltre l'abitazione � costeggiata da un torrente.
	\begin{itemize}	
		\item Circostanze che aumentano la complessit� delle misurazioni:
	\begin{itemize}
		\item \textbf{Distanza} fra sorgente e ricettore
		\item \textbf{Orografia} di tipo vallivo
		\item Presenza del \textbf{torrente}
		\item \textbf{Rumore muticiclico modulato}
	\end{itemize}
	\end{itemize}
Per questi motivi per poter caratterizzare al meglio il fenomeno rumoroso � stato necessario effettuare una misurazione in continuo per un periodo prolungato.
\end{frame}


\begin{frame}
\frametitle{Caso studio: Esposto}
\begin{center}	
	\begin{figure}
		\subfigure[Foto satellitare del luogo in esame]{
			\includegraphics[scale=0.31]{Images/Pratrivero_Manna_modificata.png}}
	\end{figure}
\end{center}
\end{frame}



\begin{frame}
\frametitle{Caso studio: Esposto}
Dopo un \textbf{sopralluogo al ricettore} si � provveduto a \textbf{pianificare} ed effettuare una serie di interventi di misura sia in esterno che in interno all'abitazione dell'esponente.\\ 
Da una prima sommaria analisi delle \textbf{misure fonometriche} corredate di\textbf{ registrazione audio} (indispensabile per le misure in continuo e prolungate) � emerso che il periodo pi� critico risultava essere quello notturno nella fascia oraria 22:00 - 06:00.\\ 
All'interno di questa fascia oraria si sono individuate le ore meno disturbate dall'attivit� casalinga dell'esponente 00:00 - 05.00.\\ 
Le successive analisi pertanto si sono concentrate in questa fascia oraria.   
\end{frame}

\begin{frame}
\frametitle{Caso studio: Esposto}
I dati raccolti sono stati scaricati dalla strumentazione in formato csv (un file ogni ora di misura). Prima dell'analisi con il software \textcolor{blue}{\textit{\textbf{R}}} � stato fatto un preliminare ascolto delle tracce audio per identificare univocamente i cicli della sorgente rumorosa.\\ 
Quindi \textcolor{blue}{\textit{\textbf{R}}} � stato utile per:
\begin{block}{}
	\begin{columns}[T]
		\begin{column}{.45\textwidth}
			\centering
			\vspace{5mm}
			\includegraphics[scale=0.10]{Images/Rlogo.png}
		\end{column}
		\begin{column}{0.55\textwidth}
			\begin{itemize}
				\item \small Unire i file orari
				\item \small Marcare tutti i cicli a $40$ $Hz$ e $315$ $Hz$ sul valore di $Leq(dB(A))$
				\item \small Unire i valori di $Leq(dB(A))$ corrispondenti ai cicli
				\item \small Ricercare eventuali toni puri
				\item \small Calcolare la media energetica oraria sul $Leq(dB(A))$ corrispondente ai cicli
				\item \small Rappresentare graficamente i risultati		
			\end{itemize}	
		\end{column}	
	\end{columns}
\end{block}
\end{frame}

\begin{frame}[fragile]
\frametitle{Caso studio: Esposto}
Nelle seguenti slides analizzeremo passo passo il codice \textcolor{blue}{\textit{\textbf{R}}} prodotto per il trattamento dei dati.
\begin{block}{Codice R}
	\begin{lrbox}{\mybox}%
		\begin{lstlisting}[basicstyle =\tiny]
			# Lettura nome file e inserimento nomi in un vettore di stringhe
			lista.files <- list.files(dirdataset)

			# Carico dataset orari (00:00 - 04:00) dei Giorni 21 - 22 e 23 settembre 2018
			for (j in 13:17) {
			assign(paste("df_", substr(lista.files[j], 1, 12), sep = ""), 
			read.table(paste(dirdataset, lista.files[j], sep = "/"), 
			skip = 6, dec = ".", sep = ";", stringsAsFactors = FALSE))
			}

			for (j in 18:length(lista.files)) {
			assign(paste("df_", substr(lista.files[j], 1, 13), sep = ""), 
			read.table(paste(dirdataset, lista.files[j], sep = "/"), 
			skip = 6, dec = ".", sep = ";", stringsAsFactors = FALSE))
			}	
		\end{lstlisting}
	\end{lrbox}{\mybox}%
	\scalebox{0.95}{\usebox{\mybox}}	
\end{block}	
\end{frame}

\begin{frame}[fragile]
\frametitle{Caso studio: Esposto}
\begin{block}{Codice R}
	\begin{lrbox}{\mybox}%
		\begin{lstlisting}[basicstyle =\tiny]
		# Operazione comune a tutti i dataset
		df_210918_00_in.name <- read.table(paste(dirdataset, lista.files[13], sep = "/"), skip = 5, nrows = 1, dec = ".", sep = ";", stringsAsFactors = FALSE)
		
		# Aggiunta delle lettere "min" dalla colonna 50 alla 87 (spettro dei minimi)
		df_210918_00_in.name[50:87] <- paste(df_210918_00_in.name[50:87], "min", sep = "")
		names(df_220918_00_out) <- df_210918_00_in.name
		# Eliminazione spazi e parentesi nel nome delle variabili
		names(df_220918_00_out) <- gsub("\\W|\\s", "", names(df_220918_00_out))
		
		# Inserimento nomi ottimizzati alle variabili 
		names(df_220918_01_out) <- names(df_220918_00_out)
		names(df_220918_02_out) <- names(df_220918_00_out)
		names(df_220918_03_out) <- names(df_220918_00_out)
		names(df_220918_04_out) <- names(df_220918_00_out)
		
		names(df_230918_00_out) <- names(df_220918_00_out)
		names(df_230918_01_out) <- names(df_220918_00_out)
		names(df_230918_02_out) <- names(df_220918_00_out)
		names(df_230918_03_out) <- names(df_220918_00_out)
		names(df_230918_04_out) <- names(df_220918_00_out)
		
		\end{lstlisting}
	\end{lrbox}{\mybox}%
	\scalebox{0.95}{\usebox{\mybox}}	
\end{block}	
\end{frame}

\begin{frame}[fragile]
\frametitle{Caso studio: Esposto}
\begin{block}{Codice R}
	\begin{lrbox}{\mybox}%
		\begin{lstlisting}[basicstyle =\tiny]
		# Funzione personalizzata per creare un sequenza: 
		# data(YYYY-MM-DD), ora(hh:mm:ss) al secondo
		time.sequence <- function(data, ora){
			start.seq <- as.POSIXct(paste(data, ora, sep = " "), 
			format("%Y-%m-%d %H:%m:%s"))
			stop.seq <- start.seq + 3599
			sequence <- seq(start.seq,
			stop.seq, by = 1)
			return(sequence)
		}
			
		# Ciclo for per creare piu sequenze orarie
		# Seguenza del 22 settembre 2018
		for (i in 0:4) {
			assign(paste0("time2209180",i), 
			time.sequence("2018-09-22", paste0("0", i, ":00:00")))
		}
		
		# Sequenza del 23 settembre 2018
		for (i in 0:4) {
			assign(paste0("time2309180",i), 
			time.sequence("2018-09-22", paste0("0", i, ":00:00")))
		}	
		\end{lstlisting}
	\end{lrbox}{\mybox}%
	\scalebox{0.95}{\usebox{\mybox}}	
\end{block}	
\end{frame}

\begin{frame}[fragile]
\frametitle{Caso studio: Esposto}
	\begin{block}{Codice R}
		\begin{lrbox}{\mybox}%
			\begin{lstlisting}[basicstyle =\tiny]
			
			#Aggiunta della colonna date ricostruita
			df_220918_00_out$date <- time22091800
			df_220918_01_out$date <- time22091801
			df_220918_02_out$date <- time22091802
			df_220918_03_out$date <- time22091803
			df_220918_04_out$date <- time22091804
			
			df_230918_00_out$date <- time23091800
			df_230918_01_out$date <- time23091801
			df_230918_02_out$date <- time23091802
			df_230918_03_out$date <- time23091803
			df_230918_04_out$date <- time23091804
			
			\end{lstlisting}			
		\end{lrbox}{\mybox}%
		\scalebox{0.95}{\usebox{\mybox}}
	\end{block}	
\end{frame}


\begin{frame}[fragile]
\frametitle{Caso studio: Esposto}
\begin{block}{Codice R}
	\begin{lrbox}{\mybox}%
		\begin{lstlisting}[basicstyle =\tiny]
		
		# Marcatura eventi fonometrici superiori a 40 dB
		df_220918_00_out$MarkerEU <- as.factor(ifelse(df_220918_00_out$`40Hz` > 42, "M", "N"))
		df_220918_01_out$MarkerEU <- as.factor(ifelse(df_220918_01_out$`40Hz` > 42, "M", "N"))
		df_220918_02_out$MarkerEU <- as.factor(ifelse(df_220918_02_out$`40Hz` > 42, "M", "N"))
		df_220918_03_out$MarkerEU <- as.factor(ifelse(df_220918_03_out$`40Hz` > 42, "M", "N"))
		df_220918_04_out$MarkerEU <- as.factor(ifelse(df_220918_04_out$`40Hz` > 42, "M", "N"))
		
		df_230918_00_out$MarkerEU <- as.factor(ifelse(df_230918_00_out$`40Hz` > 42, "M", "N"))
		df_230918_01_out$MarkerEU <- as.factor(ifelse(df_230918_01_out$`40Hz` > 42, "M", "N"))
		df_230918_02_out$MarkerEU <- as.factor(ifelse(df_230918_02_out$`40Hz` > 42, "M", "N"))
		df_230918_03_out$MarkerEU <- as.factor(ifelse(df_230918_03_out$`40Hz` > 42, "M", "N"))
		df_230918_04_out$MarkerEU <- as.factor(ifelse(df_230918_04_out$`40Hz` > 42, "M", "N"))
			
		\end{lstlisting}
	\end{lrbox}{\mybox}%
	\scalebox{0.95}{\usebox{\mybox}}	
\end{block}	
\end{frame}

\begin{frame}[fragile]
\frametitle{Caso studio: Esposto}
\begin{block}{Codice R}
	\begin{lrbox}{\mybox}%
		\begin{lstlisting}[basicstyle =\tiny]
		
		# Unione dei dataframe
		
		Unione_df_2223_09_18 <- rbind(
										f_220918_00_out,
										df_220918_01_out,
										df_220918_02_out,
										df_220918_03_out,
										df_220918_04_out,
		
										df_230918_00_out,
										df_230918_01_out,
										df_230918_02_out,
										df_230918_03_out,
										df_230918_04_out
										)
		
		\end{lstlisting}
	\end{lrbox}{\mybox}%
	\scalebox{0.95}{\usebox{\mybox}}	
\end{block}	
\end{frame}


\begin{frame}[fragile]
\frametitle{Caso studio: Esposto}
\begin{block}{Codice R}
	\begin{lrbox}{\mybox}%
		\begin{lstlisting}[basicstyle =\tiny]
		
	# Grafico: Particolare delle misure effettuate
	plot(df17072018_in$LLeq_40Hz, type = "l", ylim = c(0, 55), 
		ylab = "Livello sonoro (dB/dB(A))",
		main = "Grafico: Estratto misure pre-bonifica",
		sub = "Particolare del 17/07/2018 dalle 23:21:10 alle 23:29:18 - scansione 1 secondo")
		lines(df17072018_in$LLeq_315Hz, col = "blue")
		lines(df17072018_in$LAeq, col = "red")
		legend("bottomleft", c("Leq(dB(A))", "Leq(dB) - 40 Hz", "Leq(dB) - 315 Hz"),
		col = c("red", "black", "blue"), lty = c(1,1,1), inset = 0.01, 
		cex = 0.85, y.intersp = 0.30, x.intersp = 0.2, bty = "n", text.font = 2)
		
		\end{lstlisting}
	\end{lrbox}{\mybox}%
	\scalebox{0.95}{\usebox{\mybox}}	
\end{block}	
\end{frame}

\begin{frame}
\frametitle{Caso studio: Esposto}
\begin{center}
	
	\begin{figure}
		\subfigure[Analisi esplorativa]{
			\includegraphics[scale=0.30]{Images/Rplot_Cicli_Manna.pdf}}
	\end{figure}	
\end{center}
\end{frame}


\begin{frame}[fragile]
\frametitle{Caso studio: Esposto}
\begin{block}{Codice R}
	\begin{lrbox}{\mybox}%
		\begin{lstlisting}[basicstyle =\tiny]
		# Subsetting degli eventi fonometrici superiori a 42 dB
		Eventi_22_40Hz <- subset(Unione_df_22_09_18, MarkerEU == "M")
		plot(Eventi_22_40Hz$`40Hz`)
		
		# Isolamento dello spettro dei minimi
		Eventi_22_40Hz_min <- Eventi_22_40Hz[ , c(1, 3, 42, 50:87)]
		Eventi_22_40Hz_spettromin <- Eventi_22_40Hz[, c(50:85)]
		
		# Manipolazione dataframe gathering
		Eventi_40Hz_spettrominLong <- gather(Eventi_22_40Hz_spettromin, freq, llmin)
		
		# Creazione di una funzione personalizzata "Calcolo della media energetica 
		# (media logaritmica in base 10)"
		EnergeticMean <- function(x) {
			x <- x[!is.na(x)]
			li <- (1 / length(x))
			s <- sum(10 ^ (x / 10))
			return(round(10 * (log10(li * s)), 1))
			}
		\end{lstlisting}
	\end{lrbox}{\mybox}%
	\scalebox{0.95}{\usebox{\mybox}}	
\end{block}	
\end{frame}

\begin{frame}[fragile]
\frametitle{Caso studio: Esposto}
\begin{block}{Codice R}
	\begin{lrbox}{\mybox}%
		\begin{lstlisting}[basicstyle =\tiny]
	# Calcolo della media energetica
	Tabella_spettrominimi <- as.data.frame(as.table(
	tapply(Eventi_40Hz_spettrominLong$llmin, 
	Eventi_40Hz_spettrominLong$freq, EnergeticMean)))
	
	# Riordino delle frequenze (ordine perso "da rivedere per ottimizzare")
	# Questo passo sarebbe stato inutile se all'etichettatura delle colonne avessi
	# lasciato il punto
	freq <- c(100, 10, 10000, 12.5, 125, 1250, 12500, 160, 16, 1600, 16000, 1000, 
	200, 20, 20000,  250, 25, 2500, 2000, 31.5, 315, 3150, 400, 40, 4000, 
	500, 50, 5000, 630, 6.3, 63, 6300, 800, 80, 8, 8000)
	
	Tabella_spettrominimi <- cbind(Tabella_spettrominimi, freq)
	Tabella_spettrominimi <- Tabella_spettrominimi[order(freq), ]
	Tabella_spettrominimi <- Tabella_spettrominimi[ , -1]
	Tabella_spettrominimi <- Tabella_spettrominimi[ , c(2:1)]
	names(Tabella_spettrominimi) <- c("Hz", "LLfmin")
	#~~~~~~~~~~~~~~~~~~~~~~~~~~~~~~~~~~~~~~~~~~~~~~~~~~~~~~~~~~~~~~~~~~~~~~~~~~~~~~~
		\end{lstlisting}
	\end{lrbox}{\mybox}%
	\scalebox{0.95}{\usebox{\mybox}}	
\end{block}	
\end{frame}

\begin{frame}[fragile]
\frametitle{Caso studio: Esposto}
\textbf{ISO 226 : 1987 (E)}
\begin{equation}
L_{n} = 4.2 + \dfrac{a_{f}(L_{f} - T_{f})}{1 + b_{f}(L_{f} - T_{f})}
\end{equation}
\begin{block}{Codice R}
	\begin{lrbox}{\mybox}%
		\begin{lstlisting}[basicstyle =\tiny]
	# Grafico Isofonica vs spettro dei minimi 
	mydata <- Tabella_spettrominimi
	
	# iso	is a dataframe of values like table in ISO 226 1987 for free field
	iso <- read.table("Isofoniche.csv", sep = ";", dec = ".", header = TRUE)
	
	mydata$LLfmin <- suppressWarnings(as.numeric(as.character(mydata$LLfmin)))
	isof <- iso
	isof$LLfmin <- signif(x = mydata$LLfmin, digit = 3)
	isof$afC <- (isof$af)*(isof$LLfmin - isof$Tf)
	isof$bfC <- (isof$bf)*(isof$LLfmin - isof$Tf)
	isof$Ln <- (isof$afC)/(1 + isof$bfC) + 4.2
	isof$Ln1 <- max(isof$Ln, na.rm = TRUE)
	isof$A <- (isof$Ln1) - 4.2
	isof$Lf <- signif(x=(isof$A)/(isof$af-(isof$A)*(isof$bf)) + isof$Tf, digit = 3)
	isof <- as.data.frame(isof)
		\end{lstlisting}
	\end{lrbox}{\mybox}%
	\scalebox{0.80}{\usebox{\mybox}}	
\end{block}	
\end{frame}



\begin{frame}[fragile]
\frametitle{Caso studio: Esposto}
\textbf{ISO 226 : 1987 (E)}

\begin{block}{Codice R}
	\begin{lrbox}{\mybox}%
		\begin{lstlisting}[basicstyle =\tiny]
	# Codice per la ricerca dei toni puti
	A <- (length(isof$Hz[(which(isof$LLfmin>=isof$Lf))])==1)
	B <- isof$Hz[(which(isof$LLfmin >= isof$Lf))] == 		isof$Hz[(which(diff(isof$LLfmin) >= 5) + 1)])
	
	if (A & A %in% B){
		paste("Attenzione � presente un tono puro con freq.:", 
		isof$Hz[(which(isof$LLfmin >= isof$Lf))], "Hz", sep = " ")
	} else {print("Non sono presenti toni puri")}
	
	# Codice per il grafico spettro dei minimi Vs isofonica pi� alta
	mybar <- barplot(isof$LLfmin, names.arg = isof$Hz, col = "blue", ylim = c(0,100),
		main = "Grafico Isofonica pi� alta vs Spettro LLFmin - Campo libero", 
		xlab = "Frequenza (Hz)", ylab = "Livello sonoro (dB)", space = 0)
		lines(mybar, isof$Lf, type = "l", col = "red", lwd = 2)
		\end{lstlisting}
	\end{lrbox}{\mybox}%
	\scalebox{0.95}{\usebox{\mybox}}	
\end{block}	
\end{frame}

\begin{frame}
\frametitle{Caso studio: Esposto}
	\begin{center}
		\begin{block}{Risultato del codice R per la ricerca dei toni}
	 		\begin{figure}
	 			\subfigure[Simulazione]{
					\includegraphics[scale=0.25]{Images/plotSearchTone.pdf}}
			\end{figure}
		\end{block}
	\end{center}
\end{frame}

\begin{frame}
\frametitle{Caso studio: Esposto}
\begin{center}
	
		\begin{figure}
			\subfigure[Risultati analisi toni puri]{
				\includegraphics[scale=0.30]{Images/Rbarplot_Manna_AnalisiSpettrale.pdf}}
		\end{figure}
	
\end{center}
\end{frame}

\begin{frame}
\frametitle{Caso studio: Esposto}
\begin{center}
	
	\begin{figure}
		\subfigure[Risultati analisi spettrale]{
			\includegraphics[scale=0.30]{Images/Rplot_SpettroLLeq_Manna.pdf}}
	\end{figure}
	
\end{center}
\end{frame}

\begin{frame}
\frametitle{Caso studio: Pala Eolica Oulx}

	\begin{center}
		\begin{block}{}
	 		\begin{figure}
				\subfigure{
				\includegraphics[scale=0.38]{Images/eolico.jpg}}
			\end{figure}
		\end{block}
	\end{center}
In questo caso � stato necessario calcolare la media scalare della velocit� del vento nel \textbf{periodo diurno} (06:00 - 22.00) e nel \textbf{periodo notturno} (22:00 - 06:00) per il confronto con i dati fonometrici.
\end{frame}

\begin{frame}[fragile]
\frametitle{Caso studio: Pala eolica Oulx}
	\begin{block}{Codice R}
		\begin{lrbox}{\mybox}
			\begin{lstlisting}[basicstyle =\tiny]
			# Carico dataset
			df <- read.csv(paste(dirdataset, "Vel_VentoSCALARE.csv", sep = "/"),
						sep = ";",
						dec = ".",
						header = T,
						stringsAsFactors = F,
						na.strings = "null")
		
		# Manipolazione dataset eliminazione variabili irrilevanti per l'analisi
		# e cambio nome ad alcune variabili
		df <- df[, c(9:12)]
		names(df)[3:4] <- c("VelVento(m/s)", "Validazione")
		
		# Subsetting dati validati
		df <- subset(df, Validazione == "OK")
		\end{lstlisting}
	\end{lrbox}{\mybox}%
	\scalebox{0.95}{\usebox{\mybox}}		
\end{block}	
\end{frame}



\begin{frame}[fragile]
\frametitle{Caso studio: Pala eolica Oulx}
\begin{block}{Codice R}
	\begin{lrbox}{\mybox}
		\begin{lstlisting}[basicstyle =\tiny]
	# Inserisco i flag "giorno" (6-22) e "notte" (22-6)
	# Manipolazione data e ora
	df$date <- paste(df$data, df$ora, sep = " ")
	df$date <- dmy_hm(df$date)
	df$Ora <- hour(df$date)
	df$Giorno <- day(df$date)
	df$Mese <- month(df$date)
	
	# Inserimento flags nella nuova variabile Periodo
	df$Periodo1 <- ifelse(df$Ora <= 5, "N", "G")
	df$Periodo2 <- ifelse(df$Ora >= 22, "N", "G")
	df$Periodo <-
	ifelse(df$Periodo1 == "N" | df$Periodo2 == "N", "N", "G")
	#~~~~~~~~~~~~~~~~~~~~~~~~~~~~~~~~~~~~~~~~~~~~~~~~~~~~~~~~~~~~~~~~~~~~~~~~~~~~~~
	
	# subsetting df rispetto i flags notte "N" e giorno "G"
	df_N <- subset(df, Periodo == "N")
	df_G <- subset(df, Periodo == "G")
	#~~~~~~~~~~~~~~~~~~~~~~~~~~~~~~~~~~~~~~~~~~~~~~~~~~~~~~~~~~~~~~~~~~~~~~~~~~~~~~~
		\end{lstlisting}
	\end{lrbox}{\mybox}%
	\scalebox{0.95}{\usebox{\mybox}}		
\end{block}	
\end{frame}



\begin{frame}[fragile]
\frametitle{Caso studio: Pala eolica Oulx}
\begin{block}{Codice R}
	\begin{lrbox}{\mybox}
		\begin{lstlisting}[basicstyle =\tiny]
		# Calcolo meadia, min, massimo e deviazione standard periodo Diurno
		t_G <- tapply(df_G$`VelVento(m/s)`, list(df_G$Giorno, df_G$Mese),
		function(x) {
			c(
			MEAN = mean(x),
			MIN = min(x),
			MAX = (max(x)),
			SD = sd(x)
			)
		})
		
		# Assemblo il contenuto della lista "t_N" e lo trasformo in dataframe
		df_G_Day <- as.data.frame(do.call(rbind, t_G))
		
		# Inserisco il vettore data nel dataframe
		df_G_Day$DATA <- unique(df_G$data)
		
		# Riordino posizione variabili
		df_G_Day <- df_G_Day[, c(5, 1:4)]
		\end{lstlisting}
	\end{lrbox}{\mybox}%
	\scalebox{0.95}{\usebox{\mybox}}		
\end{block}	
\end{frame}




\begin{frame}[fragile]
\frametitle{Caso studio: Pala eolica Oulx}
\begin{block}{Codice R}
	\begin{lrbox}{\mybox}
		\begin{lstlisting}[basicstyle =\tiny]
		# Arrotondamenti ad una cifra decimale
		# Creazione funzione personalizzata per arrotondare
		round_df <- function(x, decimali) {
		# arrotonda tulle le variabili numeriche
		# x: data frame
		# decimali: numero di decimali
		numeric_columns <- sapply(x, mode) == 'numeric'
		x[numeric_columns] <-  round(x[numeric_columns], decimali)
		x
		}
		
		# Applico la funzione "round_df"
		df_G_Day <- round_df(df_G_Day, 2)
		#~~~~~~~~~~~~~~~~~~~~~~~~~~~~~~~~~~~~~~~~~~~~~~~~~~~~~~~~~~~~~~~~~~~~~~~~~~~~~~
		
		# Salvataggio files .csv elaborato nella cartella "Tabelle_Elaborate"
		write.table(
			df_G_Day,
			paste(
			dirsalvataggio, "TabGiorno_StatGiornaliereVV_2017_OULX.csv",
			sep = "/"
			),
			row.names = F,
			sep = ";",
			dec = ".")
		\end{lstlisting}
	\end{lrbox}{\mybox}%
	\scalebox{0.95}{\usebox{\mybox}}		
\end{block}	
\end{frame}

\begin{frame}[fragile]
\frametitle{Caso studio: Pala eolica Oulx}
\begin{block}{Codice R}
	\begin{lrbox}{\mybox}
		\begin{lstlisting}[basicstyle =\tiny]
		# codice per calcolo statistiche notturne a cavallo di due giorni
		# Inserimento flags alle ore notturne divise per >= 22 "N1" il resto "N2"
		df_N$PeriodoAcu <- ifelse(df_N$Ora >= 22, "N1", "N2")
		# Eliminazione variabili non necessarie
		df_N <- df_N[, c(1, 5, 6:8, 3, 4, 11:12)]
		#~~~~~~~~~~~~~~~~~~~~~~~~~~~~~~~~~~~~~~~~~~~~~~~~~~~~~~~~~~~~~~~~~~~~~~~~~~~~~~~
		#~~~~~~~~~~~~~~~~~~~~~~~~~~~~~~~~~~~~~~~~~~~~~~~~~~~~~~~~~~~~~~~~~~~~~~~~~~~~~~~
		# subsetting per l'eliminazione delle prime e ultime ore che non completano il
		# range orario da analizzare
		df_N1 <- df_N[!(df_N$data == unique((df_N$data))[1] & df_N$Ora < 22),]
		
		df_N2 <- df_N1[!(df_N1$data == unique((df_N1$data))[365] &
				 df_N1$Ora >= 22),]
		#~~~~~~~~~~~~~~~~~~~~~~~~~~~~~~~~~~~~~~~~~~~~~~~~~~~~~~~~~~~~~~~~~~~~~~~~~~~~~~~
		# ciclo che mette in lista i dataframe subsettati secondo i criteri 22-6 di due
		# giorni consecutivi
		a <- list()
		for (i in 1:length(unique(df_N2$data))) {
		a[[i]] <- subset(df_N2,
						Ora %in% c(23, 22, 0, 1, 2, 3, 4, 5) &
						df_N2$data %in% c(unique((df_N2$data))[i],
										  unique((df_N2$data))[i + 1]))
		a[[i]] <- a[[i]][!(a[[i]]$data == unique((df_N2$data))[i] &
							a[[i]]$Ora %in% c(0, 1, 2, 3, 4, 5)),]
		
		a[[i]] <- a[[i]][!(a[[i]]$data == unique((df_N2$data))[i + 1] &
							a[[i]]$Ora %in% c(23, 22)),]
		}
		\end{lstlisting}
	\end{lrbox}{\mybox}%
	\scalebox{0.75}{\usebox{\mybox}}		
\end{block}	
\end{frame}

\begin{frame}[fragile]
\frametitle{Caso studio: Pala eolica Oulx}
\begin{block}{Codice R}
	\begin{lrbox}{\mybox}
		\begin{lstlisting}[basicstyle =\tiny]
		# funzione personalizzata per il calcolo delle statistiche di base
		summary_night <- function(x) {
			c(
			MEAN = mean(x),
			MIN = min(x),
			MAX = max(x),
			SD = sd(x)
		)
		}
		
		# Applicazione funzione sulla lista contenente 364 dataframe 22-6
		b <- list()
		for (j in 1:length(a)) {
		b[[j]] <- summary_night(a[[j]][6]$`VelVento(m/s)`)
		}
		
		# Creazione del dataframe finale con i risultati
		df_nigthACU <- as.data.frame(do.call(rbind, b))
		
		# Inserimento colonna DATA
		df_nigthACU$DATA <- unique(df_N2$data)
		
		# Riarrangiamento colonne del dataframe
		df_nigthACU <- df_nigthACU[, c(5, 1, 2, 3, 4)]
		
		\end{lstlisting}
	\end{lrbox}{\mybox}%
	\scalebox{0.85}{\usebox{\mybox}}		
\end{block}	
\end{frame}

\begin{frame}[fragile]
\frametitle{Caso studio: Pala eolica Oulx}
\begin{block}{Codice R}
	\begin{lrbox}{\mybox}
		\begin{lstlisting}[basicstyle =\tiny]
		# Arrotondamenti ad n cifre decimali
		# Creazione funzione personalizzata per arrotondare
		round_df <- function(x, decimali) {
		# arrotonda tulle le variabili numeriche
		# x: data frame
		# decimali: numero di decimali
		numeric_columns <- sapply(x, mode) == 'numeric'
		x[numeric_columns] <-  round(x[numeric_columns], decimali)
		x
		}
		
		# Applico la funzione "round_df"
		df_nigthACU <- round_df(df_nigthACU, 2)
		
		
		# Salvataggio dataframe in un file csv
		write.table(
		df_nigthACU,
		paste(dirsalvataggio, "TabNotteACU_StatGiornaliereVV_2017_OULX.csv",
			sep = "/"),
			sep = ";",
			dec = ".",
			row.names = F
		)
		################################################################################
		\end{lstlisting}
	\end{lrbox}{\mybox}%
	\scalebox{0.85}{\usebox{\mybox}}		
\end{block}	
\end{frame}

\begin{frame}
\frametitle{Conclusioni}

Concludendo si pu� chiaramente affermare che l'\textbf{uso} di \textcolor{blue}{\textbf{R}} nell'ambito dell'\textbf{acustica ambientale} e pi� in generale nell'ambito dell'analisi dei \textbf{dati ambientali} sta divenendo \textbf{indispensabile} pena la \textcolor{red}{\textbf{svalutazione}} della \textbf{mole importante} di \textbf{dati} che ogni giorno, ora, minuto e secondo ARPA produce.\\

\begin{block}{Due risate...}
	\begin{figure}
		\includegraphics[scale=0.30]{Images/superman-data-scientist-graphic.jpg}
		\hspace{10mm}
		\includegraphics[scale=0.28]{Images/Sexiest-Job-of-the-21st-Century-2.jpg}
	\end{figure}
\end{block}
\end{frame}

\begin{frame}
\frametitle{Bibliografia}
\begin{itemize}
\item The R manuals - edited by the R Development Core Team.
\item R Core Team (2018). R: A language and environment for statistical computing. R Foundation for Statistical Computing, Vienna, Austria. URL: \href{https://www.R-project.org/}{https://www.R-project.org/}.
\item Manuale di acustica applicata - Renato Spagnolo Casa editrice: UTET Edizione: marzo 2004.
\end{itemize}
\end{frame}

\begin{frame}
\frametitle{Grazie per l'attenzione!}
\begin{figure}
\includegraphics[scale=0.52]{Images/wordcloud_packages.png}
\end{figure}
\end{frame}

\end{document}
