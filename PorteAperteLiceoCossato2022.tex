%~~~~~~~~~~~~~~~~~~~~~~~~~~~~~~~~~~~~~~~~~~~~~~~~~~~~~~~~~~~~~~~~~~~~~~~~~~~~~~~
%Presentazione: Corso: PORTE APERTE ALL'ARPA
%Luogo: Cossato - Acustica Ambientale
%Autore: Simone Sperotto, Pasquale Scordino
%Data: 01 marzo 2022
%~~~~~~~~~~~~~~~~~~~~~~~~~~~~~~~~~~~~~~~~~~~~~~~~~~~~~~~~~~~~~~~~~~~~~~~~~~~~~~~
\documentclass{beamer}
\usetheme{CambridgeUS}
\usecolortheme{beaver}
\usefonttheme[onlymath]{serif}

%Settaggio colori e forma degli elenchi puntati
\setbeamercolor{itemize item}{fg=darkred!80!black}
\setbeamercolor{itemize subitem}{fg=orange}
\setbeamercolor{itemize subsubitem}{fg=cyan}

\setbeamertemplate{itemize item}[triangle]
\setbeamertemplate{itemize subitem}[circle]
\setbeamertemplate{itemize subsubitem}[triangle]

%Settaggio colore testo e sfondo del titolo del blocco
\setbeamercolor{block title}{fg=white,bg=darkred!80!black}
%~~~~~~~~~~~~~~~~~~~~~~~~~~~~~~~~~~~~~~~~~~~~~~~~~~~~~~~~~~~~~~~~~~~~~~~~~~~~~~~~
%Pacchetti usati
\usepackage[italian]{babel}
\usepackage[latin1]{inputenc}
\usepackage{times}
\usepackage[T1]{fontenc}
\usepackage{graphics}
\usepackage{subfigure}
\usepackage{multimedia}
\usepackage{pifont,xcolor} % http://ctan.org/pkg/{pifont,xcolor}
\usepackage{fancybox}
\usepackage{textpos}
\usepackage{amsmath}
\usepackage{xcolor}
\usepackage[document]{ragged2e} % per giustificare il testo
%~~~~~~~~~~~~~~~~~~~~~~~~~~~~~~~~~~~~~~~~~~~~~~~~~~~~~~~~~~~~~~~~~~~~~~~~~~~~~~~~~
%Pacchetto usato per scrivere codice R in Beamer e suo settaggio
\usepackage{listings}
\usepackage{color}
\definecolor{dkgreen}{rgb}{0,0.6,0}
\definecolor{gray}{rgb}{0.5,0.5,0.5}
\definecolor{mauve}{rgb}{0.58,0,0.82}
\lstset{ %
	language=R,                     % the language of the code
	basicstyle=\footnotesize,       % the size of the fonts that are used for the code
	numbers=left,                   % where to put the line-numbers
	numberstyle=\tiny\color{gray},  % the style that is used for the line-numbers
	stepnumber=1,                   % the step between two line-numbers. If it's 1, each line
	% will be numbered
	numbersep=5pt,                  % how far the line-numbers are from the code
	backgroundcolor=\color{white},  % choose the background color. You must add \usepackage{color}
	showspaces=false,               % show spaces adding particular underscores
	showstringspaces=false,         % underline spaces within strings
	showtabs=false,                 % show tabs within strings adding particular underscores
	frame=single,                   % adds a frame around the code
	rulecolor=\color{black},        % if not set, the frame-color may be changed on line-breaks within not-black text (e.g. commens (green here))
	tabsize=2,                      % sets default tabsize to 2 spaces
	captionpos=b,                   % sets the caption-position to bottom
	breaklines=true,                % sets automatic line breaking
	breakatwhitespace=false,        % sets if automatic breaks should only happen at whitespace
	title=\lstname,                 % show the filename of files included with \lstinputlisting;
	% also try caption instead of title
	keywordstyle=\color{blue},      % keyword style
	commentstyle=\color{dkgreen},   % comment style
	stringstyle=\color{mauve},      % string literal style
	escapeinside={\%*}{*)},         % if you want to add a comment within your code
	morekeywords={*,...}            % if you want to add more keywords to the set
}

\newsavebox{\mybox}

%~~~~~~~~~~~~~~~~~~~~~~~~~~~~~~~~~~~~~~~~~~~~~~~~~~~~~~~~~~~~~~~~~~~~~~~~~~~~~~~~~

%Eventuale aggiunta immagine background

%\setbeamertemplate{background} 
%{
%	\includegraphics[width=\paperwidth,height=\paperheight]{MyBackground.jpg}
%}


%Prima slide intestazione 
\title[Porte aperte all'ARPA]{ACUSTICA AMBIENTALE - Attivit� sede Biella}

\author[S. Sperotto\hspace{0.45cm}{s.sperotto@arpa.piemonte.it}]{S. Sperotto}
\institute{ARPA Piemonte}
\date{16 marzo 2022}

\titlegraphic{

  \includegraphics[width=2cm,height=2cm,keepaspectratio]{Images/Arpacolori.jpg}%
  \hspace*{8.20cm}
  \includegraphics[width=2cm,height=2cm,keepaspectratio]{Images/logo_SNPA_COL.jpg}%

}
%~~~~~~~~~~~~~~~~~~~~~~~~~~~~~~~~~~~~~~~~~~~~~~~~~~~~~~~~~~~~~~~~~~~~~~~~~~~~~~~~~
%Inizio 
\begin{document}

\begin{frame}
\titlepage
\end{frame}

\begin{frame}
	\frametitle{Attivit� del servizio acustica ambientale}
	\begin{center}
		\begin{block}{Attivit� principali}
			\begin{itemize}
				\vspace{2mm}
				\item \textbf{Pareri tecnici:} \small Sono delle \textcolor{orange}{valutazioni tecniche} effettuate sulla documentazione (Relazioni impatto/clima acustico, misure fonometriche ecc...) presentata dai consulenti, al fine di verificarne i \textcolor{orange}{requisiti minimi di accettabilit�} e eventualmente definire delle \textcolor{orange}{prescrizioni.}.
				\vspace{3mm}
				\item \textbf{Esposti:} \small Se un cittadino ritiene di aver individuato \textcolor{orange}{violazioni delle leggi in materia di acustica ambientale}, pu� informare il \textbf{Comune} che, nell'ambito della propria attivit� di controllo, verificher� la segnalazione chiedendo, eventualmente, \underline{supporto tecnico ad ARPA}. 
				\vspace{3mm}
				\item \textbf{Monitoraggi:} \small Il monitoraggio ambientale � definito dalla European Environment Agency (EEA) come "la misurazione, valutazione e determinazione di parametri ambientali e/o di livelli di inquinamento, periodiche e/o continuate allo scopo di prevenire effetti negativi e dannosi verso l'ambiente". 
				\vspace{2mm}
			\end{itemize}	
		\end{block}
	\end{center}
\end{frame}

\begin{frame}
	\frametitle{Attivit� del servizio acustica ambientale}
	
	\begin{center}
		\begin{block}{Un p� di statistiche (-:}
			\begin{figure}
				\subfigure[]{
					\includegraphics[scale=0.25]{Images/p1_attivita.png}}
				\hspace{6mm}
				\subfigure[]{
					\includegraphics[scale=0.25]{Images/p2_attuazione.png}}
			\end{figure}
		\end{block}
	\end{center}
\end{frame}

\begin{frame}
	\frametitle{Organizzazione servizio di acustica ambientale}
		\justify	
		Presso il servizio di tutela e vigilanza della sede di Biella c'� attualmente uno specialista in ambito dell'acustica ambientale che svolge le attivit� di \textbf{redazione pareri tecnici}, \textbf{sopralluoghi e misurazioni fonometriche}, a seguito di esposto, ed effettuazione di approfondimenti attraverso l'esecuzione di \textbf{monitoraggi acustici ambientali}.
\end{frame}



\begin{frame}
	\frametitle{Esempio: Esposto}
	\begin{center}	
		\begin{figure}
			\subfigure[Foto satellitare del luogo in esame]{
				\includegraphics[scale=0.31]{Images/Pratrivero_Manna_modificata.png}}
		\end{figure}
	\end{center}
\end{frame}


\begin{frame}
\frametitle{Esempio: Esposto}
\begin{block}{Esempi elaborazioni grafiche}
	\begin{figure}
		\subfigure[]{
			\includegraphics[scale=0.19]{Images/Rplot_Cicli_Manna.pdf}}
		\subfigure[]{
			\includegraphics[scale=0.19]{Images/Rplot_SpettroLLeq_Manna.pdf}}
	\end{figure}
\end{block}
\end{frame}

\begin{frame}
\frametitle{Esempio: Esposto}
\begin{block}{Esempi elaborazioni grafiche}
	\begin{figure}
		\subfigure[]{
			\includegraphics[scale=0.25]{Images/Rbarplot_Manna_AnalisiSpettrale.pdf}}
	\end{figure}
\end{block}
\end{frame}

\begin{frame}
\frametitle{Esempio: Esposto}
	\begin{center}
		\begin{block}{Risultato ricerca dei toni - simulazione}
	 		\begin{figure}
	 			\subfigure[Simulazione]{
					\includegraphics[scale=0.25]{Images/plotSearchTone.pdf}}
			\end{figure}
		\end{block}
	\end{center}
\end{frame}

\begin{frame}
	\frametitle{Esempio: Monitoraggio}
	%\begin{block}{Foto stazioni mobili di misurazione}
		\begin{figure}[h]
			%\subfigure[]{
				\includegraphics[scale=0.045, angle=270]{Images/foto_Anteo.jpg}%}
			%\hfil
			%\subfigure[]{
			%	\includegraphics[scale=0.028]{Images/Foto_Lamarmora.jpg}}
		\end{figure}
	%\end{block}
\end{frame}

\begin{frame}
	\frametitle{Esempio: Monitoraggio}
	\begin{center}	
		\begin{figure}
			\subfigure[Foto satellitare del luogo in esame]{
				\includegraphics[scale=0.31]{Images/mappa_punti_covid_Biella1.pdf}}
		\end{figure}
	\end{center}
\end{frame}

\begin{frame}
	\frametitle{Esempio: Monitoraggio}
	\begin{block}{Elaborazioni grafiche}
		\begin{figure}
			\subfigure[]{
				\includegraphics[scale=0.28]{Images/BoxPlot_leq_festferiali_totale.pdf}}
			\subfigure[]{
				\includegraphics[scale=0.26]{Images/GraficoGiornaliero_totale.pdf}}
		\end{figure}
	\end{block}
\end{frame}

\begin{frame}
	\frametitle{Esempio: Monitoraggio}
	%\begin{block}{Elaborazioni grafiche}
		\begin{figure}
			%\subfigure[]{
				\includegraphics[scale=0.17]{Images/livelli_diurni_edifici_SP142.pdf}
			%}
			%\subfigure[]{
			%	\includegraphics[scale=0.26]{Images/GraficoGiornaliero_totale.pdf}}
		\end{figure}
	%\end{block}
\end{frame}

\begin{frame}
\frametitle{Conclusioni}

Concludendo...\\

\begin{block}{Grazie per l'attenzione e buon rumore a tutti!}
	\begin{figure}
		\includegraphics[scale=0.20]{Images/RumoreFumetto.jpeg}
		\hspace{10mm}
		\includegraphics[scale=0.5]{Images/Finale.jpeg}
	\end{figure}
\end{block}
\end{frame}


\end{document}
