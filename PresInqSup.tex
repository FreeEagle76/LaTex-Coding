\documentclass[10pt]{beamer}

\usetheme{AnnArbor}
\usepackage[italian]{babel}
\usepackage[applemac]{inputenc}
\setbeamercovered{transparent}
\pgfdeclareimage[height=1cm]{logo}{LogoArpa}
\logo{\pgfuseimage{logo}}
\title{L'INQUINAMENTO ATMOSFERICO}
\titlegraphic{\includegraphics[width=1.0in]{Inq2}}%{\pgfuseimage{logo}}
\author{Pasquale Scordino}
\subtitle{Ovvero: \textbf{COSA RESPIRIAMO?}}
\institute[ARPA Piemonte - Dip. Biella]{ARPA Piemonte - Dipartimento di Biella \\ \url{https://www.arpa.piemonte.it}}
\date{15 Marzo 2011}

\begin{document}

  \transduration{1}
  \frame{
  \transsplitverticalin
  \titlepage }
  \section*{Sommario}
\part{1a Parte: L'atmosfera}

\section{Struttura dell'atmosfera}  
  \frame{
  \transsplitverticalin
  \frametitle{L'atmosfera terrestre 1}
  
 \begin{columns}[c]  
    \column{2.3in}
     \framebox{\includegraphics[width=2.3in]{Immagine1.jpg}} 
    \column{1.9in}
  \begin{itemize}\pause
   \item[]{La troposfera, analizzando le masse d'aria in movimento, puo essere suddivisa in:} 
  \end{itemize}      
    \begin{itemize}
      \item {\alert{Strato limite planetario} (PBL)}
      \item {\alert{Troposfera libera}}
    \end{itemize}
  \end{columns}
  }

\subsection{L'inversione termica}
\frame{
  \transsplitverticalin
  \frametitle{L'atmosfera terrestre 2}
\textbf{Foto dalla alta valle Po alle ore 17.50 del 5 Agosto 2005 - Inversione termica.}
\vspace{10}
\begin{center}
     \framebox{\includegraphics[width=3in]{inversione.jpg}}  
\end{center}
  } 

\frame{
  \transsplitverticalin
  \frametitle{L'atmosfera terrestre 3}
\textbf{Rappresentazione del fenomeno dell'inversione termica.}
\vspace{10}
\begin{center}
     \framebox{\includegraphics[width=3in]{PBLInversione.jpg}}   
\end{center}
  } 

\section{Composizione chimica dell'atmosfera}
 \frame{
  \transsplitverticalin
  \frametitle{La composizione chimica dell'atmosfera}
\begin{columns}[c]  
    \column{2.5in}
     \framebox{\includegraphics[width=2in]{TabcompAtm.jpg}} 
    \column{2in}
      \framebox{\includegraphics[width=1.5in]{tortacompatm.jpg}} 
  \end{columns}
  } 

\subsection{Chimica della troposfera}
  \frame{
    \transsplitverticalin
    \frametitle{La chimica della troposfera 1}
La chimica dell'atmosfera \'e guidata da una specie chimica chiamata spazzino atmosferico: \alert{Il radicale ossidrile.}\\\pause
\vspace{10}
In bassa troposfera e in assenza di ossidi di azoto e composti organici, 
sia di origine naturale che antropica, avvengono le seguenti reazioni chimiche di fotolisi dell'ozono:  
\vspace{20}
\begin{columns}[c]  
    \column{2in}
      $O_{3} + \alert{h\nu} \rightarrow O_{2} + O(^{1}D)$\\
    \vspace{2}
      $O(^{1}D) + M \rightarrow O(^{3}P) + M$\\
    \vspace{10}
      $O(^{3}P) + O_2 + M \rightarrow O_3 + M$\\
    \vspace{2}
      $O(^{1}D) + H_2 O \rightarrow 2OH^{\cdotp}$\\
    \column{2in}
      $(\lambda \leq 335 nm)$\\
    \vspace{5}
      $(M = N_2 , O_2)$\\
    \vspace{10}
      $(M = aria)$
    \vspace{5}
\end{columns} 
  } 

\frame{
    \transsplitverticalin
    \frametitle{La chimica della troposfera 2}
In atmosfera povera di $NO_x$ ma con la presenza di $VOC_s$ (Composti organici volatili) di origine naturale le
reazioni chimiche che avvengono sono le seguenti:\\\pause
\vspace{20}
\begin{columns}[c]  
    \column{2in}
      $HO_{2}^{\cdotp} + O_{3} \rightarrow OH^{\cdotp} + 2O_{2}$\\
    \vspace{2}
       $HO_{2}^{\cdotp} + HO_{2}^{\cdotp} \rightarrow H_{2}O_{2} + O_{2}$\\
    \vspace{2}
      $RO_{2}^{\cdotp} + HO_{2}^{\cdotp} \rightarrow ROOH + O_{2}$\\
    \vspace{2}
      $H_{2}O_{2} + \alert{h\nu} (\lambda < 370 nm) \rightarrow 2OH^{\cdotp} $
    \column{2in}
\setbeamercolor{postit}{fg=black,bg=yellow}
\setbeamercolor{postut}{fg=black,bg=cyan}
\begin{beamerboxesrounded}[upper=postit ,lower=postut ,shadow=true]{}
I perossidi organici formatisi si depositano sulle superfici o si disciolgono 
nell'aereosol acquoso. L'acqua ossigenata, invece, fotolizza dando altri radicali ossidrili.
\end{beamerboxesrounded}
    \vspace{5}
\end{columns}
  } 

\subsection{Chimica della stratosfera}
\frame{
    \transsplitverticalin
    \frametitle{La chimica della stratosfera}
In Stratosfera l'ozono si forma attraverso il ciclo di Chapman:\\\pause
\vspace{20}
\begin{columns}[c]  
    \column{2in}
      $O_{2} + \alert{h\nu} (\lambda \leq 290 nm) \rightarrow 2O^{\cdotp}$\\
    \vspace{2}
       $O^{\cdotp} + O_{2} + M \rightarrow O_{3}$\\
    \vspace{2}
      $O^{\cdotp} + O_{3} \rightarrow 2O_{2}$\\
    \vspace{2}
      $O_{3} + \alert{h\nu} \rightarrow O^{\cdotp} + O_{2}$\\
    \column{2in}
\setbeamercolor{postit}{fg=black,bg=yellow}
\setbeamercolor{postut}{fg=black,bg=cyan}
\begin{beamerboxesrounded}[upper=postit ,lower=postut ,shadow=true]{}
Il ciclo di Chapman \'e stato disturbato da specie chimiche persistenti di origine antropica
che arrivate in stratosfera sono state fotolizzate ed il loro prodotto ha perturbato il ciclo.
\end{beamerboxesrounded}
    \vspace{5}
\end{columns}
  } 

\frame{
  \transsplitverticalin
  \frametitle{La chimica della stratosfera}
\textbf{Assorbimento spettrale dell'atmosfera.}
\vspace{10}
\begin{center}
     \framebox{\includegraphics[width=3in]{spettro.jpg}}   
  \end{center}
  } 

\part{2a Parte: L'inquinamento atmosferico}
\section{L'inquinamento atmosferico}
  \frame{
    \transsplitverticalin
    \frametitle{L'inquinamento atmosferico 1}
\begin{center}
Si definisce \alert{inquinamento atmosferico} ogni modificazione dell'aria atmosferica, dovuta all'introduzione
nella stessa di una o pi\'u sostanze  in quantit\'a e con caratteristiche tali da ledere o da 
costituire un pericolo per la salute umana o per la qualit\'a dell'ambiente oppure tali da ledere i
 beni materiali o compromettere gli usi legittimi dell'ambiente.
\end{center}
  } 

 \frame{
    \transsplitverticalin
    \frametitle{L'inquinamento atmosferico 2}
L'introduzione di sostanze di origine antropica, ma anche naturale, oltre una certa soglia provoca lo squilibrio dei naturali 
cicli chimici.
\begin{center}
\framebox{\includegraphics[width=3.3in]{Immagine2.jpg}} 
\end{center}
  } 

\subsection{Gli inquinanti atmosferici}
\frame{
    \transsplitverticalin
    \frametitle{L'inquinamento atmosferico 3}
\textbf{Le sostanze inquinanti possono essere classificate in diversi modi:}\pause
\begin{center}
  \begin{itemize}
   \item Secondo la loro origine.
      \begin{itemize}\pause
       \item Naturale (incendi boschivi, eruzioni vulcaniche, emissioni $VOC_s$ dalla vegetazione ecc...).
       \item Antropica (attivit\'a industriali, traffico veicolare, riscaldamento civile e industriale ecc...).
      \end{itemize}
   \item Secondo la classe chimica di appartenenza.
\begin{itemize}\pause
       \item Sostanze organiche volatili (questa classe \'e numerosa \'e pu\'o essere suddivisa ulteriormente...). 
       \item Metalli pesanti.
       \item Ossidi inorganici.
\end{itemize}
   \item Secondo il loro comportamento/evoluzione in atmosfera.
\begin{itemize}\pause
       \item Primari (emessi direttamente dalle sorgenti senza subire trasformazioni chimiche in atmosfera).
       \item Secondari (originano da trasformazioni chimiche in atmosfera con la fondamentale partecipazione di precursori).
      \end{itemize}
\end{itemize}
\end{center}
  } 

\subsection{L'inquinamento Estivo}
\frame{
    \transsplitverticalin
    \frametitle{L'inquinamento atmosferico 4}
    \framesubtitle{Inquinamento tipico primaverile - estivo - 1}
Nell'inquinamento tipico primaverile estivo la fa da padrone l'ozono troposferico:
\vspace{10}
\begin{itemize}\pause
 \item L'ozono \'e un inquinante di tipo secondario.\pause
 \item Per la sua formazione necessita di precursori quali: $VOC_s$, $NO_x$ e radiazione solare.\pause
 \item E' un forte ossidante tossico per la salute e dannoso per i materiali plastici.
\end{itemize}
  } 

\frame{
    \transsplitverticalin
    \frametitle{L'inquinamento atmosferico}
    \framesubtitle{Inquinamento tipico primaverile - estivo - 2}
La formazione fotochimica dell'ozono troposferico \'e influenzata pesantemente dalla presenza 
di $NO_x$ e $VOC_s$. Le reazioni chimiche principali sono le seguenti:
\vspace{20}
\begin{columns}[c]  
    \column{2.5in}
      $NO_{2} + \alert{h\nu} (\lambda \leq 420 nm) \rightarrow NO + O(^{3}P)^{\cdotp}$\\
    \vspace{2}
       $O(^{3}P)^{\cdotp} + O_{2} + M \rightarrow O_{3} + M$\\
    \vspace{2}
      $NO + O_{3} \rightarrow NO_2 + O_{2}$\\
    \vspace{10}
      $HO_{2}^{\cdotp} + NO \rightarrow OH^{\cdotp} + NO_{2}$\\
    \vspace{2}
      $RO_{2}^{\cdotp} + NO \rightarrow RO^{\cdotp} + NO_{2}$\\
    \column{2in}
\setbeamercolor{postit}{fg=black,bg=yellow}
\setbeamercolor{postut}{fg=black,bg=cyan}
\begin{beamerboxesrounded}[upper=postit ,lower=postut ,shadow=true]{}
La presenza di $VOC_s$ rende il ciclo di fotolisi dell' $NO_2$ instabile con 
formazione netta di ozono.
\end{beamerboxesrounded}
    \vspace{5}
\end{columns}
  }  

\subsection{L'inquinamento invernale}
\frame{
    \transsplitverticalin
    \frametitle{L'inquinamento atmosferico}
    \framesubtitle{Inquinamento tipico autunno - invernale - 1}
Nell'inquinamento tipico autunno - invernale i principali inquinanti incriminati sono gli 
aerosol atmosferici e il biossido di azoto ($NO_2$). Le caratteristiche principali degli aerosol sono le seguenti:
\vspace{10}
\begin{itemize}\pause
 \item Sono inquinanti sia di origine primaria che secondaria.\pause
 \item Sono eterogenei nella loro composizione e forma/grandezza.\pause
 \item La loro tossicit\'a varia a seconda della composizione chimico-fisica.\pause
 \item Sono classificabili/campionabili secondo il loro diametro aerodinamico.
\end{itemize}
  } 

\frame{
    \transsplitverticalin
    \frametitle{L'inquinamento atmosferico}
    \framesubtitle{Inquinamento tipico autunno - invernale - 2}
\textbf{Il Particolato Atmosferico PM10 e PM2.5}
\vspace{10}
\begin{columns}[c]  
    \column{2in}
     \framebox{\includegraphics[width=2in]{Immagine8.jpg}} 
    \column{2in}
      \framebox{\includegraphics[width=1.5in]{Immagine9.jpg}} 
  \end{columns}
  }

\frame{
    \transsplitverticalin
    \frametitle{L'inquinamento atmosferico}
    \framesubtitle{Inquinamento tipico autunno - invernale - 3}
\textbf{Il Particolato Atmosferico PM - Uno sguardo al microscopio:}
\vspace{10}
\begin{columns}[c]  
    \column{1.5in}
     \framebox{\includegraphics[width=1.5in]{polvere1.jpg}} 
	\begin{center}
	 \caption{PM10}
	\end{center}
    \column{1.5in}
      \framebox{\includegraphics[width=1.5in]{polvere2.jpg}}
	\begin{center}
	 \caption{PM2.5}
	\end{center}
    \column{1.5in}
      \framebox{\includegraphics[width=1.5in]{polvere3.jpg}} 
	\begin{center}
	 \caption{PM1}
	\end{center}
  \end{columns}
  }

\frame{
    \transsplitverticalin
    \frametitle{L'inquinamento atmosferico}
    \framesubtitle{Inquinamento tipico autunno - invernale - 4}
\textbf{Il Particolato Atmosferico PM - Uno sguardo al microscopio rispetto alcune sorgenti:}
\vspace{10}
\begin{columns}[c]  
    \column{1.5in}
     \framebox{\includegraphics[width=1.5in]{polvere4.jpg}}
	\begin{center}
	 \caption{metano}
	\end{center}	
    \column{1.5in}
      \framebox{\includegraphics[width=1.5in]{polvere5.jpg}}
	\begin{center}
	 \caption{legna}
	\end{center}
    \column{1.5in}
      \framebox{\includegraphics[width=1.5in]{polvere6.jpg}} 
	 \begin{center}
	 \caption{gasolio}
	\end{center}
  \end{columns}
  }

\section{Monitoraggio dell'inquinamento atmosferico}
\frame{
    \transsplitverticalin
    \frametitle{Monitoraggio dell'inquinamento atmosferico}
Il monitoraggio degli inquinanti presenti in atmosfera si effettua utilizzando dalle pi\'u classiche 
alle pi\'u moderne tecniche di chimica analitica.\\ ARPA Piemonte gestisce, dal punto di vista tecnico scientifico, 
numerose stazioni di rilevamento degli inquinanti che contengono al loro interno gli strumenti atti alla quantificazione di ogni
inquinante.
\vspace{10}
\begin{columns}[c]  
    \column{1.5in}
     \framebox{\includegraphics[width=1.3in]{MMBi2.jpg}} 
    \column{1.5in}
      \framebox{\includegraphics[width=1.3in]{MMBi.jpg}} 
  \end{columns}
}

\frame{
    \transsplitverticalin
    \frametitle{Monitoraggio dell'inquinamento atmosferico - 2}
Il Dipartimento Arpa di Biella dispone di 5 stazioni fisse e 1 laboratorio mobile per un totale di
circa 25 parametri chimici monitorati e 260000 dati l'anno prodotti.
\vspace{10}
\begin{columns}[c]  
    \column{1.5in}
     \framebox{\includegraphics[width=1.5in]{MappaP.jpg}} 
    \column{1.5in}
      \framebox{\includegraphics[width=1.3in]{STABi1.jpg}} 
  \end{columns}
}

\frame{
    \transsplitverticalin
    \frametitle{Monitoraggio dell'inquinamento atmosferico - 3}
\textbf{I principali inquinanti chimici monitorati sono i seguenti:}
\vspace{10}
\begin{itemize}\pause
 \item Monossido di carbonio ($CO$).
 \item Ossidi di azoto ($NO$ - \alert{$NO_2$}).
 \item Biossido di zolfo ($SO_2$).
 \item Ozono (\alert{$O_3$}).
 \item Materiale particolato \alert{$PM10$} e \alert{$PM2.5$}.
    \begin{itemize}\pause
     \item Metalli pesanti ($As$, $Cd$, $Ni$, $Pb$).
     \item Idrocarburi policiclici aromatici ($Benzo[a]pirene$).
    \end{itemize}
 \item Idrocarburi aromatici ($BTX$).
\end{itemize} 
}

\subsection{Monitoraggio del $CO$}
\frame{
    \transsplitverticalin
    \frametitle{Monitoraggio dell'inquinamento atmosferico - 4}
\begin{center}
 \textbf{Monossido di carbonio - $CO$}
\end{center}
\vspace{5}
Gas tossico che si forma durante la combustione incompleta dei combustibili usati 
negli autoveicoli e in generale in qualsiasi processo combustivo in difetto di ossigeno.\\
\vspace{5}
Il valore limite \'e di $10 mg/m^{3}$ come media massima giornaliera su 8 ore calcolata ogni ora sulla base delle otto
ore precedenti.\\
\vspace{5}
Il suo monitoraggio viene effettuato utilizzando la tecnica di assorbimento IR non dispersivo.
Una sorgente di radiazione IR policromatica viene fatta passare attraverso il campione e una cella di riferimento
contenente un gas che non assorbe nell'IR. Un rivelatore a valle misura la differenza di riscaldamento che producono.
}

\frame{
    \transsplitverticalin
    \frametitle{Monitoraggio dell'inquinamento atmosferico - 5}
\textbf{Schema strumentale per il monitoraggio del monossido di carbonio - $CO$}
\begin{center}
\begin{columns}[c]  
    \column{2in}
     \framebox{\includegraphics[width=2in]{SchemaCO.jpg}}  
\end{columns}
\end{center}
}

\subsection{Monitoraggio del $SO_2$}
\frame{
    \transsplitverticalin
    \frametitle{Monitoraggio dell'inquinamento atmosferico - 6}
\begin{center}
 \textbf{Biossido di zolfo - $SO_2$}
\end{center}

Si forma per combustione di materiali contenenti zolfo, quali i combustibili fossili come il 
carbone, nafta e gasolio. Nell'aria si combina facilmente con l'acqua atmosferica dando inizialmente 
acido solforoso e  successivamente, per ossidazione lenta, acido solforico 
(uno dei responsabili delle piogge acide).\\

I valori limite dell' $SO_2$ sono diversi:
\begin{itemize}
 \item $350 \mu g/m^{3}$ come valore limite orario da non superare pi\'u di 18 volte l'anno.
 \item $125 \mu g/m^{3}$ come valore limite giornaliero da non superare pi\'u di 3 volte l'anno.
 \item $500 \mu g/m^{3}$ soglia di allarme - media oraria per tre ore consecutive.
\end{itemize}

Il suo monitoraggio viene effettuato utilizzando la tecnica fluorimetrica, sfruttando 
la caratteristica dell' $SO_2$ di emettere, se irradiata con una radiazione monocromatica opportuna, una radiazione a 
lunghezza d'onda diversa\\ da quella di incidenza.
}

\frame{
    \transsplitverticalin
    \frametitle{Monitoraggio dell'inquinamento atmosferico - 7}
\textbf{Schema strumentale per il monitoraggio del Biossido di zolfo - $SO_2$}
\begin{center}
\begin{columns}[c]  
    \column{3in}
     \framebox{\includegraphics[width=3in]{StrumSO2.jpg}}  
\end{columns}
\end{center}
}

\subsection{Monitoraggio degli idrocarburi aromatici}
\frame{
    \transsplitverticalin
    \frametitle{Monitoraggio dell'inquinamento atmosferico - 8}
\begin{center}
 \textbf{Idrocarburi aromatici - $BTX$}
\end{center}
\vspace{5}
Gli Idrocarburi aromatici sono una classe chimica di composti caratterizzati dalla presenza 
di un anello benzenico.\\
La loro concentrazione in atmosfera \'e direttamente correlabile al traffico veicolare: infatti il benzene \'e 
un componente importante delle benzine.\\
\vspace{5}
Il valore limite \'e di $5 \mu g/m^{3}$ come media annuale.\\
\vspace{5}
Il suo monitoraggio viene effettuato utilizzando la tecnica gascromatografica con 
rivelatore PID a fotoionizzazione.
}

\frame{
    \transsplitverticalin
    \frametitle{Monitoraggio dell'inquinamento atmosferico - 9}
\textbf{Schema strumentale per il monitoraggio degli idrocarburi aromatici - $BTX$}
\vspace{10}
\begin{columns}[c]  
    \column{2in}
     \framebox{\includegraphics[width=2.0in]{schemagc.jpeg}}  
    \column{2in}
     \framebox{\includegraphics[width=2.0in]{PID.jpeg}}  
\end{columns}
}

\frame{
    \transsplitverticalin
    \frametitle{Monitoraggio dell'inquinamento atmosferico - 10}
\textbf{Strumenti e colonna per il monitoraggio degli idrocarburi aromatici - $BTX$}
\vspace{10}
\begin{columns}[c]  
    \column{1.5in}
     \framebox{\includegraphics[width=1.5in]{gc1.jpeg}}  
    \column{1.5in}
     \framebox{\includegraphics[width=1.5in]{colonnagc.jpeg}}
    \column{1.5in}
     \framebox{\includegraphics[width=1.5in]{GCgramma.jpg}}
\end{columns}
}

\frame{
    \transsplitverticalin
    \frametitle{Monitoraggio dell'inquinamento atmosferico - 11}
\textbf{Alcuni dati sugli idrocarburi aromatici - $BTX$}
\vspace{5}
\begin{columns}[c]  
    \column{2in}
     \framebox{\includegraphics[width=1.9in]{SerieTempB2010.jpeg}}  
    \column{2in}
     \framebox{\includegraphics[width=1.9in]{MediaBBar.jpeg}}
\end{columns}
}

\frame{
    \transsplitverticalin
    \frametitle{Monitoraggio dell'inquinamento atmosferico - 12}
\textbf{Alcuni dati sugli idrocarburi aromatici - $BTX$}
\vspace{5}
\begin{columns}[c]  
    \column{2in}
     \framebox{\includegraphics[width=1.9in]{GiornoMedioBiLamB2010.jpeg}}  
    \column{2in}
     \framebox{\includegraphics[width=1.9in]{GiornoMedioCossB2010.jpeg}}
\end{columns}
}

\frame{
    \transsplitverticalin
    \frametitle{Monitoraggio dell'inquinamento atmosferico - 13}
\textbf{Alcuni dati sugli idrocarburi aromatici - $BTX$}
\vspace{5}
\begin{center}
\begin{columns}[c]  
    \column{3in}
     \framebox{\includegraphics[width=3in]{ValBenz.jpg}}  
\end{columns}
\end{center}
}

\subsection{Monitoraggio del $O_3$}
\frame{
    \transsplitverticalin
    \frametitle{Monitoraggio dell'inquinamento atmosferico - 14}
\begin{center}
 \textbf{Ozono - $O_3$}
\end{center}
\vspace{5}
L'ozono \'e un gas pungente che non ha sorgenti dirette significative, ma si produce 
all'interno di un ciclo di reazioni fotochimiche.\\
\vspace{5}
I valori limite sono i seguenti:\\
\begin{itemize}
 \item $120 \mu g/m^{3}$ come valore bersaglio da non superare pi\'u di 25 volte l'anno 
(max media mobile su 8 ore).
 \item $180 \mu g/m^{3}$ come soglia di informazione (media oraria).
 \item $240 \mu g/m^{3}$ come soglia di allarme per tre ore consecutive (media oraria).
\end{itemize}
\vspace{5}
Il suo monitoraggio viene effettuato utilizzando la tecnica spettrofotometrica.
}

\frame{
    \transsplitverticalin
    \frametitle{Monitoraggio dell'inquinamento atmosferico - 15}
\textbf{Schema strumentale per il monitoraggio dell'Ozono - $O_3$}
\vspace{10}
\begin{center}
\begin{columns}[c]  
    \column{3in}
     \framebox{\includegraphics[width=3.0in]{ozonoanalisi.jpg}}  
\end{columns}
\end{center}
}

\frame{
    \transsplitverticalin
    \frametitle{Monitoraggio dell'inquinamento atmosferico - 16}
\textbf{Alcuni dati sull'Ozono - $O_3$}
\vspace{5}
\begin{columns}[c]  
    \column{2in}
     \framebox{\includegraphics[width=1.9in]{SerieTempO32010.jpeg}}  
    \column{2in}
     \framebox{\includegraphics[width=1.9in]{NSupO3Bar.jpeg}}
\end{columns}
}

\frame{
    \transsplitverticalin
    \frametitle{Monitoraggio dell'inquinamento atmosferico - 17}
\textbf{Alcuni dati sull'Ozono - $O_3$}
\vspace{5}
\begin{columns}[c]  
    \column{2in}
     \framebox{\includegraphics[width=1.9in]{GiornoMedioBiStuO32010.jpeg}}  
    \column{2in}
     \framebox{\includegraphics[width=1.9in]{GiornoMedioCossO32010.jpeg}}
\end{columns}
}

\frame{
    \transsplitverticalin
    \frametitle{Monitoraggio dell'inquinamento atmosferico - 18}
\textbf{Alcuni dati sull'Ozono - $O_3$}
\vspace{5}
\begin{center}
\begin{columns}[c]  
    \column{2.5in}
     \framebox{\includegraphics[width=2.4in]{Labplot1.jpeg}}  
\column{2.5in}
     \framebox{\includegraphics[width=2.4in]{Labplot2.jpeg}}  
\end{columns}
\end{center}
}

\subsection{Monitoraggio degli $NO_x$}
\frame{
    \transsplitverticalin
    \frametitle{Monitoraggio dell'inquinamento atmosferico - 19}
\begin{center}
 \textbf{Biossidi di azoto - $NO_x$}
\end{center}
\vspace{5}
Gli $NO_x$ comprendono principalmente gli $NO$ e gli $NO_2$ sono composti che si formano nei processi di combustione ad alta
temperatura e sono i principali attori nella formazione del particolato e dello smog fotochimico.\\
\vspace{5}
I valori limite sono:
\begin{itemize}
 \item $200 \mu g/m^{3}$ come valore limite orario da non superare 
pi\'u di 18 volte l'anno.
 \item $40 \mu g/m^{3}$ come valore limite annuale.
  \item $400 \mu g/m^{3}$ soglia di allarme - media oraria per tre ore consecutive.
\end{itemize}
\vspace{5}
Il loro monitoraggio viene effettuato utilizzando la tecnica per chemiluminescenza.
}

\frame{
    \transsplitverticalin
    \frametitle{Monitoraggio dell'inquinamento atmosferico - 20}
\textbf{Schema strumentale per il monitoraggio degli ossidi di azoto - $NO_x$}
\vspace{15}
\begin{columns}[c]  
    \column{2in}
     \framebox{\includegraphics[width=2in]{StrumNOx.jpg}}
    \column{2in}
$NO + O_3 \longrightarrow NO_2^\ast + O_2$
\vspace{5}
$NO_2^\ast \longrightarrow NO_2 + h\nu$
\vspace{5}
$2NO_2 + Mo \longrightarrow NO + MoO_3 $
\end{columns}
}

\frame{
    \transsplitverticalin
    \frametitle{Monitoraggio dell'inquinamento atmosferico - 21}
\textbf{Alcuni dati sugli ossidi di azoto - $NO_x$}
\vspace{5}
\begin{columns}[c]  
    \column{2in}
     \framebox{\includegraphics[width=1.9in]{SerieTempNO2010.jpeg}}  
    \column{2in}
     \framebox{\includegraphics[width=1.9in]{SerieTempNO22010.jpeg}}
\end{columns}
}


\frame{
    \transsplitverticalin
    \frametitle{Monitoraggio dell'inquinamento atmosferico - 22}
\textbf{Alcuni dati sugli ossidi di azoto - $NO_x$}
\vspace{5}
\begin{center}
\begin{columns}[c]  
    \column{3in}
     \framebox{\includegraphics[width=3in]{ValNOx.jpg}}  
\end{columns}
\end{center}
}

\subsection{Monitoraggio del particolato atmosferico}
\frame{
    \transsplitverticalin
    \frametitle{Monitoraggio dell'inquinamento atmosferico - 23}
\begin{center}
 \textbf{Particolato atmosferico - $PM$}
\end{center}
Il particolato atmosferico \'e una sospensione complessa di particelle solide, liquide o miste in equilibrio
 con la fase gassosa. Le sorgenti sono sia primarie che secondarie e le principali sono il traffico veicolare
, gli impianti di riscaldamento sia civili che industriali. Il particolato con diametro aerodinamico inferiore 
a $10 \mu m$ \'e di particolare interesse tossicologico.\\\pause
\vspace{5}
I valori limite per il PM10 sono:\\\pause
\begin{itemize}
 \item $50 \mu g/m^{3}$ come valore limite giornaliero (media giornaliera).
 \item $40  \mu g/m^{3}$ come valore limite annuale.
 \item $35$ numero massimo di superamenti del limite giornaliero in un anno.
\end{itemize}
\vspace{5}
Il suo monitoraggio viene effettuato in diversi modi:\\\pause
\begin{itemize}
 \item Tecniche in continuo.
  \begin{itemize}\pause
    \item TEOM - microbilancia a smorzamento di oscillazione.
    \item Attenuazione raggi $\beta$.
    \item Scattering radiazione elettromagnetica.
\end{itemize}
 \item Tecnica gravimetrica (metodo ufficiale).
\end{itemize}
}

\frame{
    \transsplitverticalin
    \frametitle{Monitoraggio dell'inquinamento atmosferico - 24}
\textbf{Strumenti per il monitoraggio in continuo del particolato atmosferico - $PM$}
\vspace{10}
\begin{columns}[c]  
    \column{1.5in}
     \framebox{\includegraphics[width=1.3in]{TestaPMamericana.jpg}}  
    \column{1.5in}
     \framebox{\includegraphics[width=1.1in]{Teom.jpg}}  
\end{columns}
}

\frame{
    \transsplitverticalin
    \frametitle{Monitoraggio dell'inquinamento atmosferico - 25}
\textbf{Strumenti per il monitoraggio gravimetrico del particolato atmosferico - $PM$}
\vspace{10}
\begin{columns}[c]  
    \column{1.5in}
     \framebox{\includegraphics[width=1.5in]{Skypost.jpg}}  
    \column{1.5in}
     \framebox{\includegraphics[width=1.5in]{SchemaTestaPM.jpg}}
    \column{1.5in}
     \framebox{\includegraphics[width=1.5in]{flussoPesoPM.jpg}}
\end{columns}
}

\frame{
    \transsplitverticalin
    \frametitle{Monitoraggio dell'inquinamento atmosferico - 26}
\textbf{La speciazione del particolato atmosferico - $PM$}\\
\vspace{10}
Successivamente al campionamento e alla determinazione gravimetrica del PM, i campioni vengono trattati e analizzati per determinare
il loro contenuto in metalli, IPA e diversi altri componenti chimici.\\
\vspace{10}
\begin{columns}[c]  
    \column{2.5in}
     \framebox{\includegraphics[width=2.5in]{filtroaliquote.jpeg}}  
\end{columns}
}

\subsection{La speciazione del particolato atmosferico}
\frame{
    \transsplitverticalin
    \frametitle{Monitoraggio dell'inquinamento atmosferico - 27}
\textbf{La speciazione del particolato atmosferico - $PM$}\\
\vspace{10}
La speciazione chimica del particolato atmosferico si effettua con diverse techiche di chimica analitica 
previo adeguato trattamento del campione:\\
\vspace{10}
\begin{itemize}
 \item Metalli pesanti (ICP/MS, ICP/AES).
 \item Idrocarburi policiclici aromatici (GC/MS, HPLC).
 \item Anioni e cationi (CI)
 \item Sostanze organiche (CG/MS, HPLC).
\end{itemize}
}

\frame{
    \transsplitverticalin
    \frametitle{Monitoraggio dell'inquinamento atmosferico - 28}
\textbf{La speciazione del particolato atmosferico $PM$ - Determinazione dei metalli}\\
\vspace{10}
Spettrometria di massa al plasma accoppiato induttivamente - ICP/MS
\vspace{10}
\begin{columns}[c]  
    \column{2.5in}
     \framebox{\includegraphics[width=2.5in]{icp-ms.jpg}}  
    \column{2in}
     \framebox{\includegraphics[width=2in]{ICPMS.jpeg}}
\end{columns}
}

\frame{
    \transsplitverticalin
    \frametitle{Monitoraggio dell'inquinamento atmosferico - 29}
\textbf{La speciazione del particolato atmosferico $PM$ - Determinazione delle Sostanze organiche}\\
\vspace{10}
Gascromatografia accoppiata alla spettrometria di massa - GC/MS
\vspace{10}
\begin{columns}[c]  
    \column{1.5in}
     \framebox{\includegraphics[width=1.5in]{AnalFrazOrganica.jpg}}  
    \column{1.5in}
     \framebox{\includegraphics[width=1.5in]{gc1.jpeg}}
\end{columns}
}

\frame{
    \transsplitverticalin
    \frametitle{Monitoraggio dell'inquinamento atmosferico - 30}
\textbf{La speciazione del particolato atmosferico $PM$ - Determinazione dei cationi e anioni}\\
\vspace{10}
Cromatografia ionica - CI
\vspace{5}
\begin{columns}[c]  
    \column{2in}
     \framebox{\includegraphics[width=2in]{schemaCI.jpeg}}  
    \column{2in}
     \framebox{\includegraphics[width=2in]{CI.jpg}}
\end{columns}
}

\frame{
    \transsplitverticalin
    \frametitle{Monitoraggio dell'inquinamento atmosferico - 31}
\textbf{Alcuni dati sul particolato atmosferico - $PM$}
\vspace{5}
\begin{columns}[c]  
    \column{2in}
     \framebox{\includegraphics[width=1.9in]{MediaPM10Bar.jpeg}}  
    \column{2in}
     \framebox{\includegraphics[width=1.9in]{NSupPM10Bar.jpeg}}
\end{columns}
}

\frame{
    \transsplitverticalin
    \frametitle{Monitoraggio dell'inquinamento atmosferico - 32}
\textbf{Alcuni dati sul particolato atmosferico - $PM$}
\vspace{5}
\begin{columns}[c]  
    \column{2in}
     \framebox{\includegraphics[width=1.9in]{GiornoMedioBiStuPM10Teom2010.jpeg}}  
    \column{2in}
     \framebox{\includegraphics[width=1.9in]{GiornoMedioCossPM10Teom2010.jpeg}}
\end{columns}
}

\frame{
    \transsplitverticalin
    \frametitle{Monitoraggio dell'inquinamento atmosferico - 33}
\textbf{Alcuni dati sul particolato atmosferico - $PM$}
\vspace{5}
\begin{center}
\begin{columns}[c]  
    \column{3in}
     \framebox{\includegraphics[width=3in]{ValPM10Teom.jpg}}  
\end{columns}
\end{center}
}


\frame{
    \transsplitverticalin
    \frametitle{Monitoraggio dell'inquinamento atmosferico - 34}
\textbf{Alcuni dati sul particolato atmosferico - $PM$}
\vspace{5}
\begin{center}
\begin{columns}[c]  
    \column{2in}
     \framebox{\includegraphics[width=2in]{BiPM102008.jpeg}}  
    \column{1.5in}
     \framebox{\includegraphics[width=1.9in]{sat.jpeg}}
\end{columns}
\end{center}
}

\frame{
    \transsplitverticalin
    \frametitle{Monitoraggio dell'inquinamento atmosferico - 35}
\textbf{Alcuni dati sul particolato atmosferico - $PM$}
\vspace{5}
\begin{center}
\begin{columns}[c]  
    \column{2in}
     \framebox{\includegraphics[width=2in]{meteo11.jpg}}  
    \column{1.9in}
     \framebox{\includegraphics[width=1.9in]{velvento.jpeg}}
\end{columns}
\end{center}
}

\frame{
    \transsplitverticalin
    \frametitle{Monitoraggio dell'inquinamento atmosferico - 36}
\textbf{Alcuni dati sul particolato atmosferico - $PM$}
\vspace{5}
\begin{center}
\begin{columns}[c]  
    \column{2in}
     \framebox{\includegraphics[width=2in]{meteo6.jpg}}  
    \column{2in}
     \framebox{\includegraphics[width=2in]{meteo1.jpg}}
\end{columns}
\end{center}
}

\subsection{Tossicologia degli inquinanti atmosferici}
 \frame{
    \transsplitverticalin
    \frametitle{Effetti dell'inquinamento sulla salute umana}
Il monitoraggio degli inquinanti atmosferici ha come fine principale la salvaguardia della salute umana.
Alla fine dell'ottocento la sensibilit\'a verso l'inquinamento atmosferico ha iniziato ad aumentare significativamente a 
seguito di vari episodi eclatanti di morte causata da aria insalubre:\\\pause
\begin{itemize}
 \item 1873/1963 - Londra
  \begin{itemize}\pause
   \item 1873 - 500 persone morte.
   \item 1880 - 1000 persone morte.
   \item 1892 - 1000 persone morte.
   \item 1948 al 1962 ci furono diversi episodi ma il pi\'u disastroso f\'u nel dicembre del 1952 (4000 mila morti in 5 giorni)
($SO_2$).
  \end{itemize}
  \item 1984 - Disastro di Bhopal pi\'u di 2000 morti e 300000 feriti ($CH_3CN$).
  \item 1948 - Citt\'a di Donora USA 20 morti in 14 ore ($F, S, CO,$ metalli pesanti).
  \item 1930 - Valle di Meuse diverse migliaia di casi di attacchi polmonari acuti e 60 morti ($SO_2$).
  \item 1950 - Messico/Poza Rica 320 persone ospedalizzate e 22 morti\\ in 3 ore ($H_2S$).
\end{itemize}   
}
 
\frame{
    \transsplitverticalin
    \frametitle{Effetti dell'inquinamento sulla salute umana - 2}
\textbf{Londra - Dicembre 1952}
\vspace{5}
\begin{center}
\begin{columns}[c]  
    \column{2in}
     \framebox{\includegraphics[width=1.5in]{london-smog-1.jpg}}  
    \column{2in}
     \framebox{\includegraphics[width=1.5in]{smog-data.jpg}}
\end{columns}
\end{center}
}

\frame{
    \transsplitverticalin
    \frametitle{Effetti dell'inquinamento sulla salute umana - 3}
\textbf{Cenni di tossicologia degli inquinanti} 
\begin{itemize}
 \item Monossido di carbonio - $CO$
  \begin{itemize}\pause
   \item Il $CO$ si lega all'emoglobina, con una affinit\'a 200 volte superiore all'ossigeno, formando la carbossiemoglobina che
impedisce il traspoto di ossigeno ai vari distretti corporei.
  \end{itemize}
  \item Biossido di zolfo - $SO_2$
  \begin{itemize}\pause
   \item L'$SO_2$ \'e un gas molto irritante; causa, anche a basse concentrazioni, irritazione alle vie 
respiratorie. Esposizioni croniche comportano faringiti e affaticamento/disturbi 
all'apparato respiratorio.
  \end{itemize}
  \item Benzene
  \begin{itemize}\pause
   \item Il benzene \'e una sostanza classificata cancerogena per l'uomo.
  \end{itemize}
  \item Biossido di azoto $NO_2$.
  \begin{itemize}\pause
   \item L'$NO_2$ \'e un gas tossico e irritante per le mucose ed \'e responsabile di patologie a 
carico del sistema respiratorio.
  \end{itemize}
 \end{itemize}   
}

\frame{
    \transsplitverticalin
    \frametitle{Effetti dell'inquinamento sulla salute umana - 4}
\textbf{Cenni di tossicologia dell'ozono}
\vspace{5}
\begin{center}
\begin{columns}[c]  
    \column{2.5in}
     \framebox{\includegraphics[width=2.4in]{Immagine4.jpg}}  
    \column{2in}
\setbeamercolor{postit}{fg=black,bg=yellow}
\setbeamercolor{postut}{fg=black,bg=cyan}
\begin{beamerboxesrounded}[upper=postit ,lower=postut ,shadow=true]{}
Concentrazioni relativamente basse di ozono provocano effetti quali irritazione alla gola, alle vie
respiratorie e bruciore agli occhi; concentrazioni pi\'u elevate alterano le funzioni respiratorie.
\end{beamerboxesrounded}
\end{columns}
\end{center}
}

\frame{
    \transsplitverticalin
    \frametitle{Effetti dell'inquinamento sulla salute umana - 5}
\textbf{Cenni di tossicologia del particolato atmosferico}
\vspace{5}
\begin{center}
\begin{columns}[c]  
    \column{2.5in}
     \framebox{\includegraphics[width=2.5in]{DeadPM1.jpeg}}  
\end{columns}
\end{center}
}

\frame{
    \transsplitverticalin
    \frametitle{Effetti dell'inquinamento sulla salute umana - 6}
\textbf{Cenni di tossicologia del particolato atmosferico}
\vspace{5}
\begin{center}
\begin{columns}[c]  
    \column{2in}
     \framebox{\includegraphics[width=2in]{polmoni.jpg}}
  \column{2in}
     \framebox{\includegraphics[width=1.9in]{TossicologiaBaP.jpg}}
\end{columns}
\end{center}
}

\frame{
    \transsplitverticalin
    \frametitle{Effetti dell'inquinamento sulla salute umana - 7}
\textbf{Cenni di tossicologia del particolato atmosferico}
\vspace{5}
\begin{center}
\begin{columns}[c]  
    \column{2in}
     \framebox{\includegraphics[width=1.8in]{CardioPM.jpeg}}  
\end{columns}
\end{center}
}

\part{3a Parte: Conclusioni}
\frame{
    \transsplitverticalin
    \frametitle{Conclusioni}
\begin{center}
\alert{L'atmosfera \'e un sistema complesso e come tutti i sistemi complessi \'e molto delicata. \'E nostro compito preservarla per noi 
e per i nostri figli.\\ Non dimentichiamoci che i nostri comportamenti sono artefici del nostro destino e del destino della terra.}   
\end{center}
\begin{center}
\begin{columns}[c]  
    \column{1.8in}
     \framebox{\includegraphics[width=1.8in]{carov.jpg}} 
\end{columns}
\end{center}
}

\frame{
    \transsplitverticalin
    \frametitle{Bibliografia}
\begin{itemize}
  \item Roger Atkinson - Atmospheric chemistry of $VOC_s$ and $NO_x$ - Atmospheric Environment 34 (2000) 2063-2101 
  \item H.R. Anderson - Air pollution and mortality: A history - Atmospheric Environment 43 (2009) 142-152
  \item Z. Boris - Air pollution and Cardiovascular Injury - Journal of American College of Cardiology 52-9 (2008)
  \item Pitts - Chemistry of the upper and lower atmosphere - Academic Press 2000
  \item Casarett & Doll's - Tossicologia 
  \item D.lgs 13 agosto 2010 n. 155  
\end{itemize}     
}

\frame{
    \transsplitverticalin
    \frametitle{Fine}
\begin{columns}[c]  
    \column{1in}
     \framebox{\includegraphics[width=1in]{Ubuntu.jpeg}} 
    \column{1in}
     \framebox{\includegraphics[width=1in]{Latex.jpeg}}
\end{columns}
\begin{center}
  \begin{LARGE}  
    \textbf{Grazie per l'attenzione!!!}
  \end{LARGE}
\end{center} 
\begin{columns}[c]  
    \column{1in}
     \framebox{\includegraphics[width=1in]{R.jpeg}} 
      \column{1in}
     \framebox{\includegraphics[width=1in]{LABplot.jpg}}
\end{columns}    
}
\end{document}
