\documentclass{beamer}
\usetheme{PaloAlto}
\usecolortheme{spruce}
\usepackage[italian]{babel}
\usepackage[latin1]{inputenc}
\usepackage{times}
\usepackage[T1]{fontenc}
\usepackage{graphics}
\usepackage{multimedia}
\usepackage{pifont,xcolor} % http://ctan.org/pkg/{pifont,xcolor}
\usepackage{fancybox}
%\usepackage{animate}
%\usepackage{xmpmulti}

\makeatletter
\newcommand\insertlogoii{}
\newcommand\logoii[1]{\renewcommand\insertlogoii{#1}}
\setbeamertemplate{headline}
{%
	\begin{beamercolorbox}[wd=\paperwidth]{frametitle}
		\ifx\beamer@sidebarside\beamer@lefttext%
		\else%
		\hfill%
		\fi%
		\ifdim\beamer@sidebarwidth>0pt%  
		\usebeamercolor[bg]{logo}%
		\vrule width\beamer@sidebarwidth height \beamer@headheight%
		\hskip-\beamer@sidebarwidth%
		\hbox to \beamer@sidebarwidth{\hss\vbox to
			\beamer@headheight{\vss\hbox{\color{fg}\insertlogo}\vss}\hss}%
		\hfill%
		\vrule width\beamer@sidebarwidth height \beamer@headheight%
		\hskip-\beamer@sidebarwidth%
		\hbox to \beamer@sidebarwidth{\hss\vbox to
			\beamer@headheight{\vss\hbox{\color{fg}\insertlogoii}\vss}\hss}%
		\else%
		\vrule width0pt height \beamer@headheight%  
		\fi%
	\end{beamercolorbox}
}
\makeatother

% Codice per cambiare colore al tema di sfondo
\setbeamercolor{block title}{use=structure,fg=black,bg=green!80!black}
\setbeamercolor{block body}{use=structure,fg=black,bg=white!20!white}

% Codice per cambiare il colore agli elenchi puntati
\definecolor{myblue}{RGB}{0,29,119}
\newcommand{\itemcolor}[1]{% Update list item colour
\renewcommand{\makelabel}[1]{\color{#1}\hfil ##1}}

\title{Inquinamento Acustico}
\subtitle{Introduzione}
\author{P. Scordino, S. Sperotto}
\logo{\includegraphics[height = 0.7cm]{images/Arpacolori.jpg}}
\logoii{\includegraphics[height = 0.9cm]{images/logo_SNPA_COL.jpg}}

\begin{document}
	
	\begin{frame}
		\maketitle
	\end{frame}


	\begin{frame}
		\frametitle{Il suono - 1}
\begin{center}
	Il \textbf{suono} \textit{(dal latino sonus)} � la sensazione data dalla vibrazione di un corpo in oscillazione. Tale vibrazione, che si propaga nell'aria o in un altro mezzo elastico, raggiunge l'apparato uditivo dell'orecchio che, tramite un complesso meccanismo interno, crea una sensazione "uditiva" correlata alla natura della vibrazione
\end{center}
	\end{frame}


	\begin{frame}
		\frametitle{Il suono - 2}
		\begin{center}
			\includegraphics[height = 4.5 cm]{images/SuonoProp2.png}
		\end{center}
	\end{frame}

	\begin{frame}
		\frametitle{Il suono - 3}
		\begin{center}
			\includegraphics[height = 4.5 cm]{images/Range.jpeg}
		\end{center}
	\end{frame}

	
	\begin{frame}
		\frametitle{Il rumore - 1}
		\centering{
			\includegraphics[height = 7 cm]{images/RumoreFumetto.jpeg}
		}
	\end{frame}


	\begin{frame}
		\frametitle{Il rumore - 2}
\begin{center}
	Il \textbf{rumore} viene definito come una somma di oscillazioni irregolari, intermittenti o statisticamente casuali. Dal punto di vista fisiopatologico, facendo riferimento all'impatto sul soggetto che lo subisce, il rumore pu� essere meglio definito come un suono non desiderato e disturbante. 
\end{center}
	\end{frame}

	\begin{frame}
\frametitle{Esempi di rumori}
\begin{columns}[T]
	\begin{column}{.5\textwidth}
		\begin{block}{Rumori}		
			\begin{itemize}
				\item \movie[externalviewer, autostart]{Rumore armonico}{sounds/Rumore_armonico_10.mp3}
				\item \movie[externalviewer, autostart]{Rumore bianco}{sounds/Rumore_Bianco.mp3}
				\item \movie[externalviewer, autostart]{Rumore rosa}{sounds/Rumore_Rosa.mp3}
				\item \movie[externalviewer, autostart]{Rumore marrone}{sounds/Rumore_Marrone.mp3}
			\end{itemize}
		\end{block}
				
	\end{column}
\end{columns}
\end{frame}

	\begin{frame}
		\frametitle{Esempi di rumori monotonali}
		\begin{columns}[T]
			\begin{column}{.5\textwidth}
\begin{block}{Toni puri}		
	\begin{itemize}
		\item \movie[externalviewer, autostart]{Suono a 50 Hz}{sounds/50hz.mp3}
		\item \movie[externalviewer, autostart]{Suono a 100 Hz}{sounds/100hz.mp3}
		\item \movie[externalviewer, autostart]{Suono a 200 Hz}{sounds/200hz.mp3}
		\item \movie[externalviewer, autostart]{Suono a 1000 Hz}{sounds/1000hz.mp3}
		\item \movie[externalviewer, autostart]{Suono a 10000 Hz}{sounds/10000hz.mp3}
	\end{itemize}
\end{block}

\begin{block}{}		
	\includegraphics[height = 2.5 cm]{images/ToniPuri.jpg}
\end{block}

\end{column}
\end{columns}
	\end{frame}
	
	\begin{frame}
		\frametitle{La misura del rumore}
		\centering{
		\includegraphics[height = 7 cm]{images/Rumori.jpeg}
		}
	\end{frame}

	\begin{frame}
		\frametitle{Il sensore umano per il rumore - 1}
		\centering{
		\includegraphics[height = 7 cm]{images/Orecchio.jpeg}
		}
	\end{frame} 

	\begin{frame}
		\frametitle{Il sensore umano per il rumore - 2}
		\centering{
		\includegraphics[height = 7 cm]{images/EarAnatomy_InternalEar.png}
		}
	\end{frame}

	\begin{frame}
		\frametitle{Il sensore umano per il rumore - 3}
		\centering{
		\includegraphics[height = 7 cm]{images/VieAcustiche.jpeg}
		}
	\end{frame}

	\begin{frame}
\frametitle{Effetti dell'inquinamento acustico - 1}
\begin{block}{Cosa et al. 1990}		
	\begin{itemize}
		\item[-] \textbf{effetti di danno}, vale a dire di alterazioni non reversibili o non completamente reversibili, obiettivabili dal punto di vista clinico e/o anatomopatologico;
		\item[-] \textbf{effetti di disturbo}, cio� di alterazioni temporanee delle condizioni psicofisiche del soggetto, che siano chiaramente obiettivabili, determinando effetti fisiopatologici ben definiti;
		\item[-] \textbf{sensazione di disturbo e fastidio} genericamente intesa \textit{(annoyance)}.
	\end{itemize}
\end{block}

\end{frame}

\begin{frame}
\frametitle{Effetti dell'inquinamento acustico - 2}

\begin{block}{Tratto da Job 1995, in Porter, Berry \& Flindell, 1998}		
\begin{center}
		\includegraphics[height = 6 cm]{images/SoundHealth.png}
\end{center}
\end{block}


\end{frame}


	\begin{frame}
		\frametitle{Conclusioni}
		\begin{LARGE}
		\textcolor{red}{Buon rumore a tutti!!!} 
		\end{LARGE}
		\centering{
		\includegraphics[height = 6 cm]{images/Finale.jpeg}
		}
	\end{frame}

\end{document}


